\section{Stationary Extremes}\label{sec:statio}

So far, we considered the maximum $X_{(n)}=\max_{1\leq i\leq n}X_i$ being composed of independent random variables only. 
 Now, we are interested by modeling 
$X^*_{(n)}=\max_{1\leq i\leq n}X^*_i$ where $\{X^*_i\}$ will now denote a \emph{stationary} 
sequence of $n$ random variables sharing the same marginal df $F$ as the sequence $\{X_i\}$ of independent random variables.

\begin{definition}[Stationary process] We say that the sequence $\{X_i\}$ of n random variables is (strongly) \emph{\textbf{stationary}} if, for $h\geq 0$ and $n\geq 1$, the distribution of the lagged random vector $(X_{1+h},\dots,X_{n+h})$ does not depend on h.
\end{definition}
It corresponds to physical processes whose stochastic properties are homogeneous but which may be dependent. There exist other formulations of \emph{stationarity} but we stick to this general definition.  

This dependence can take many forms and hence we need to relax the independence condition.
Let $F_{i_1,\dots,i_p}(u_1,\dots,u_p):=\text{Pr}\{X_{i_1}\leq 
u_1,\dots,X_{i_p}\leq u_p\}$ denote the joint df of 
$X_{i_1},\dots,X_{i_p}$ for any arbitrary positive integers $(i_1,\dots,i_p)$.

\begin{definition}[$D(u_n)$ dependence condition from \cite{leadbetter_extreme_1974}] 
	Let $\{u_n\}$ be a sequence of real numbers. We say that the \emph{ \boldsymbol {$D(u_n)$} \textbf{condition}} holds if for any set of integers $i_1<\dots<i_p$ and $j_1<\dots<j_q$ such that $j_1-i_p>\ell$, we have that 
	
	\begin{equation}
	|F_{i_1,\dots,i_p,j_1,\dots,j_q}(u_n,\dots,u_n;u_n,\dots,u_n)-F_{i_1,\dots,i_p}(u_n,\dots,u_n)\cdot F_{j_1,\dots,j_q}(u_n,\dots, u_n)|\leq \beta_{n,\ell}\ ,
	\end{equation}
	where $\beta_{n,\ell}$ is nondecreasing and  $\displaystyle{\lim_{n \to \infty}}\beta_{n,\ell_n}=0$ for some sequence $\ell_n=o(n)$.
\end{definition}
This condition ensures that, when the sets of variables are separated by a relatively short distance, typically $s_n=o(n)$, the long-range dependence between such events is limited in a sense that is sufficiently close to zero to have no effect on the limit extremal laws.
This result is remarkable in the sense that, provided a series has limited long-range dependence at extreme levels (i.e., where $D(u_n)$ condition holds), maxima of stationary series follow the same distributional limit laws as those of independent series.% The two following theorems will show the main implications.


\begin{theorem}[Limit distribution of maxima under $D({u_n})$, \cite{leadbetter_extreme_1974}]
	Let $\{X^*_i\}$ be a stationary sequence of $n$ iid random variables. If there exists sequences of constants $\{a_n>0\}$ and $\{b_n\}$ such that $D(u_n)$ condition holds with $u_n=a_nx+b_n$ for every real $x$, and	
	\begin{equation}
	\text{Pr}\{X^*_{(n)}\leq u_n\}\longrightarrow G^*(x), \ \ \ \ \ \ \ \, \ \ n\rightarrow\infty,
	\end{equation}
	where $G^*$ is a non-degenerate df, then $G^*$ is a member of the GEV family as presented in \hyperref[extthm]{Theorem \textbf{\ref{extthm}}}.
	
\end{theorem}


\begin{theorem}[\citet{leadbetter_extremes_1983}]\label{thm:statio2}
	Let $\{X^*_i\}$ be a stationary sequence and let $\{X_i\}$ be a iid sequence of $n$ random variables. We have under regularity conditions, 
	\begin{equation*}
	\text{\emph{Pr}}\big\{a_n^{-1}(X_{(n)}-b_n)\leq x\big\}\longrightarrow G(x), \ \ \ \ \ \ \ \ \ \ \ n\rightarrow \infty,
	\end{equation*}
	for normalizing sequences $\{a_n>0\}$ and $\{b_n\}$, where $G$ is non-degenrate, if and only if 
	
	\begin{equation*}\label{extremindex}
	\text{\emph{Pr}}\big\{a_n^{-1}(X^*_{(n)}-b_n)\leq x \big\}\longrightarrow G^*(x), \ \ \ \ \ \ \ \ \ \ \  n\rightarrow\infty,
	\end{equation*}
	where $G^*$ is the limiting df coming from a stationary process, defined by
	
	\begin{equation}\label{extindex}
	G^*(x)=G^{\theta}(x),
	\end{equation}
	for some constant $\theta\in (0,1]$ called the \emph{\textbf{extremal index}}.
	
\end{theorem}
It is evident from (\ref{extindex}) that the maximum of a stationary series will have a tendency to decrease.

\subsection{The extremal index}
The \emph{extremal index} is an important indicator quantifying the extent of extremal dependence, that is the degree at which the assumption of independence is violated. From (\ref{extindex}), it is clear that $\theta=1$ lead to an independent process, but the converse does not hold. The case $\theta= 0$ will not be considered as it is too "far" from independence and brings problems. Moreover, results of \hyperref[thm:statio2]{Theorem \textbf{\ref{thm:statio2}}} would not hold.

Formally, it can be defined as

\begin{equation}\label{exc}
\theta=\displaystyle{\lim_{n \to \infty}}\text{Pr}\Big\{\max(X_2,\dots,X_{p_n})\leq u_n\ | \ X_1\geq u_n\Big\},
\end{equation}
in the POT approach, where $p_n=o(n)$ and the sequence $u_n$ is such that Pr$\big\{X_{(n)} \leq u_n\big\}$ converges.
Hence, $\theta$ can be thought for example as the probability that an exceedance over a high threshold is the final element in a \textit{cluster of exceedances}.%, concept that we will define defined in the following subsection.


\addcontentsline{toc}{subsubsection}{Clusters of exceedances}
\subsubsection*{Clusters of exceedances}

From (\ref{exc}) and in a POT context, extremes have the tendency to occur in clusters whose \emph{mean cluster size} is $\theta^{-1}$ at the limit. Equivalently, $\theta^{-1}$ can be viewed as the factor with which the mean distance between cluster is increased.
This problem of temporal dependence make inference based on the likelihood invalid.
We name two methods that can be used to circumvent this issue : 

\begin{itemize}
	\item \textbf{Filtering out} an (approximate) independent sequence of threshold exceedances.
	\item \textbf{Declustering} : compute the maximum value in each cluster and then we model these clusters maximums as independent GP random variable. In this approach, we remove temporal dependence but we do not estimate it.
	However, information is discarded and this could be a substantial loss in meteorological applications, for instance to determine heat or cold waves.
\end{itemize}


\addcontentsline{toc}{subsubsection}{New parameters}
\subsubsection*{New parameters}
 When $0<\theta\leq 1$, we have from \hyperref[thm:statio2]{Theorem \textbf{\ref{thm:statio2}}} that $G^*$ is an EV distribution but with different scale and location parameters than $G$. If we note by ($\mu^*,\sigma^*,\xi^*$) the parameters pertaining to $G^*$ and those from $G$ kept in the usual way, we have the following relationships, when $\xi\neq 0$

\begin{equation}
\mu^* = \mu-\sigma\xi^{-1}(1-\theta^{\xi}), \ \ \ \   \ \text{and} \ \quad \sigma^*=\sigma\theta^{\xi}.
\end{equation}
In the Gumbel case ($\xi=0$), we simply have $\sigma^*=\sigma$ and $\mu^*=\mu+\log\theta$.
The fact that $\xi^*=\xi$ induce that the two distributions $G^*$ and $G$ will have the same form, following \hyperref[thm:statio2]{Theorem \textbf{\ref{thm:statio2}}}.




\addcontentsline{toc}{subsubsection}{Return levels}
\subsubsection*{Return levels}

Because of clustering, notion of return levels is more complex and the dependence appear in the definition of return levels for excess models :

\begin{equation}\label{eq:rlstatio}
r_m = u + \sigma\xi^{-1}\Big[(m\zeta_u\theta)^{\xi}-1\Big].
\end{equation}
Hence, we see that ignoring dependence will lead to an overestimation of return levels. For example, we have that :

\begin{itemize}
	\item If $\theta=1$, then the \textit{100-year-event} has probability 0.368 of not appearing in the next 100 years.
	\item If $\theta=0.1$, the event has probability of 0.904 of not appearing in the next 100 years.
\end{itemize}
Return levels will be redefined in \hyperref[sec:returnlvlnstatio]{Section \textbf{\ref{sec:returnlvlnstatio}}} for the stationary case.

\subsection{Modelling in Block Maxima}

With dependent series, modeling by means of GEV as in 
\hyperref[sec::1]{Chapter \textbf{\ref{sec::1}}} can be used in 
a similar way since the shape parameter $\xi^*$ will remain invariant. The difference is that the effective number of block maxima $n^*=n\theta$ will be reduced and hence convergence in \hyperref[extthm]{extremal Theorem \textbf{\ref{extthm}}} will be slower. 
Indeed, approximation is expected to be poorer and this will be exacerbated with increased levels of dependence in the series.
Efforts must be made to either try to increase $n$ for example by reducing the blog length, or make sure the model fit is convincing with diagnostic tools presented in \hyperref[sec:diag]{Section \textbf{\ref{sec:diag}}}.


\section{Non-Stationary Extremes}\label{nstatio}

Whereas \hyperref[sec:statio]{previous Section} relaxed the first "i" of the "iid" assumption made during  \hyperref[sec::1]{Chapter \textbf{\ref{sec::1}}} and \textbf{\ref{sec::2}} by allowing temporal dependence under stationary process, this section will now tackle the "\textbf{id}" part, i.e. assumption that the observations are \textbf{i}dentically \textbf{d}istributed. 
The stationarity assumption is not likely to hold for climatological data such as temperatures. For instance, the most obvious departure from stationarity is the presence seasonal patterns as seen in Figure \ref{fig:violin_density} with higher spread in spring or in autumn for example. Seasonal concerns should disappear for very high thresholds in excess models but this is not a valid argument since the number of data would become very small. For a sufficiently large block size in block maxima, meteorological seasons should not be an issue. 

Furthermore, the aim of this thesis will focus on the modeling of the possible trend in order to assess the climate warming. 
Our modeling will hence more focus on the analysis of different parametrizations for the mean by allowing the location parameter $\mu$ to vary with time. Even if it seems less interesting, we will also allow the scale parameter to vary in order to check if the annual maxima have varying spread over time. We will avoid to vary $\xi$ with time in order to stay in the same EV family of distributions.


There are no new general theory a for nonstationary processes in previous section  and we will then use a pragmatic approach of combining standard EV models with statistical modeling.
We will only discuss the approach in block maxima but the extrapolation to POT is straightforward for inference, model comparison and diagnostics.


\subsection{Block-Maxima}\label{sec:gev_nonstatio}

As we continue to consider yearly blocks, we do only face nonstationary concerns for the trend which could probably be imputed to the Global Warming. 
The evidence of seasonality arising when we decrease the block lengths is an issue in block maxima. For example if one consider daily maximum temperatures, seasonality will be present.
 However, we loose information by taking yearly blocks only as we do not use all the information since at least one half of the daily temperatures from October to March will not be used. To overcome this issue, dataset could be divided and different models can be applied on sub-blocks of the data, for example on the month July and August and fitting a GEV model with block length of 62 days and do similar analysis with other months. This will provide us a set of GEV models that will  describe different aspects of the process. In fact, this method would lead to some kind of a "manual" nonstationary GEV modeling.


\subsubsection*{Inference for GEV}

In a nonstationary context, ML is preferred for its adaptability to changes in model structure. In a general setting, we let a nonstationary GEV model describe the distribution $Z_t$ for $t=1,\ldots,m$ :

\begin{equation}\label{eq:gevnsta}
Z_t\sim \text{GEV}\big(\mu(t), \sigma(t),\xi(t)\big),
\end{equation}
where each of $\mu(t),\sigma(t), \xi(t)$ are expressed as 
\begin{equation}\label{eq:linknsta}
\theta(t)=b\big(X^{'}\beta\big),
\end{equation}
for a specified inverse link function $b(\cdot)$ where $\theta$ denotes either $\mu,\sigma$ or $\xi$ and where $\beta$ denote the complete vector of parameters. In our example, $Z_t$ will describe the annual maximum temperature of year $t$ for $m=116$ years.
As already stated, it will not be recommended to allow $\xi$ to vary with time. Examples of parametric expressions from (\ref{eq:linknsta}) will be given in \hyperref[sec:comp0]{Section \textbf{\ref{sec:comp0}}}.

If $g\big(z_t; \mu(t),\sigma(t),\xi(t)\big)$ denotes the GEV density (Table \ref{tab:gevdens}) with parameters $\mu(t),\sigma(t),\xi(t)$ evaluated at $z_t$, the log-likelihood of the model (\ref{eq:gevnsta}) is, provided $\xi(t)\neq0 \ \forall t$,
\begin{equation}\label{eq:loglikgevnonstatio}
\begin{aligned}
\ell(\beta)& = \sum_t^m\log g\big(z_t; \mu(t),\sigma(t),\xi(t)\big)\\ & = -\sum_t^m\Bigg\{ \log\sigma(t)+\big[1+\xi^{-1}(t)\big]\log\bigg[1+\xi(t)\cdot\sigma^{-1}(t)\cdot\Big(z_t-\mu(t)\Big)\bigg]_+ \\ & \qquad\qquad\qquad\qquad\qquad\qquad\qquad\quad +  \bigg[1+\xi(t)\cdot\sigma^{-1}(t)\cdot\Big(z_t-\mu(t)\Big)\bigg]_+^{-\xi^{-1}(t)} \Bigg\},
\end{aligned}
\end{equation}
where the notation  $y_+(t)=\text{max}\big\{y(t),0\big\}$ holds for all $t$. The parameters $\mu(t),\sigma(t),\xi(t)$ are replaced by their respective expressions from (\ref{eq:linknsta}). If $\xi(t)=0$ for any $t$, we replace the likelihood by using the limit $\xi(t)\to0$ in (\ref{eq:loglikgevnonstatio}) as in Table \ref{gevdens}. 
Numerical techniques are then used to maximize (\ref{eq:loglikgevnonstatio}) in order to yield the MLE of $\beta$ and evaluate standard errors.  



\subsubsection*{Model Comparisons}

In order to compare our models, that is for example to check whether a trend is statistically significant, or if the nonstationnary models provide an improvement over the simpler (stationary) model, we will use two techniques :

\begin{enumerate}
	\item The \emph{deviance statistic} which is defined as 
	\begin{equation}
	D = 2\big\{\ell_1(\mathcal{M}_1)-\ell_0(\mathcal{M}_0)\big\},
	\end{equation}
	for two nested models $\mathcal{M}_0\subset \mathcal{M}_1$, where $\ell_1(\mathcal{M}_1)$ and $\ell_0(\mathcal{M}_0)$ are the maximized log-likelihoods (\ref{eq:loglikgevnonstatio}) under models $\mathcal{M}_1$ and $\mathcal{M}_0$ respectively.
	Asymptotically, the distribution of $D$ is $\chi^2_k$ with $k$ degrees of freedom (df) representing the difference of parameters between model $\mathcal{M}_1$ and $\mathcal{M}_0$. Comparisons of $D$ with the $\chi^2_k$ critical value will guide our decision.  
	
	\item\label{aicbic} It is sometimes preferable to rely on other criterion, for example when the number of models to be compared is large or their construction is not straightforward. We will make use of the \emph{Bayesian Information Criterion} (BIC) and  the \emph{Akaike Information Criterion (corrected) } ($\text{AIC}_{\text{c}}$). For $n$ observations and $p$ parameters, 
we define 
	\begin{equation}\label{aicbic}
\text{BIC} = -2\ell+ p\log(n), \qquad\quad \text{AIC}_{\text{c}}=-2\ell+2p+\frac{2p(p+1)}{n-p-1}.
	\end{equation}
	 Used by \citet{cannon_flexible_2010}, these two criterion both have a likelihood term which represent the quality of fit of the model and a term which penalizes the complexity of the model represented by its number of parameters to be estimated. These two criterion are advised for small samples and to prevent overfitting (i.e., fitting to noise instead of the true underlying process). BIC will penalize more heavily models that are more complex.
\end{enumerate}
The basic principle of parsimony holds for both methods. In the first method, it is incorporated in the statistical test since the critical value $\chi^2_k$ will increase with $k$, the difference of complexity of two models. In the second method, it is directly included in the criterion since the partial derivative with $p$ is positive for both criterion(\ref{aicbic}) and these are to be minimized. %econd approach could lead to more subjectivity. 

We will use the first technique in \hyperref[sec:comp0]{Section \textbf{\ref{sec:comp0}}} to make successive comparisons of parametric models that we propose for the location parameter. We will then summarize all the results in Table \ref{tab:comp_mod} in the \hyperref[sec:nnxp]{following Section} by means of BIC and $\text{AIC}_{\text{c}}$, taking all relevant models into account.


\subsubsection*{Model Diagnostics}


When the "best" model is selected following the method, it is still necessary to assess that the model fits well enough the data at hand that we can infer conclusions about some aspect of the population. 
We will use tools seen in the independent (or stationary) case, the quantile and the probability plots presented in \hyperref[app:qqpp]{Appendix \textbf{\ref{app:qqpp}}}. From the inhomogeneous distribution across years, it is needed to standardize the data. For example, when model (\ref{eq:gevnsta}) is estimated, the \emph{standardized variables} $\tilde{Z}_t$ are 

\begin{equation}
\tilde{Z}_t=\hat{\xi}^{-1}(t)\cdot\log \bigg\{1+\hat{\xi}(t)\cdot \hat{\sigma}^{-1}(t)\cdot\Big(Z_t-\hat{\mu}(t)\Big)\bigg\},
\end{equation}
each having standard Gumbel distribution (\ref{gumb}). This yield the \emph{residual probability plot} :

\begin{equation}
\bigg\{\Big(i/(m+1),\  \exp(-e^{-\tilde{z}_{(i)}})\Big) \ : \ i=1,\dots,m\bigg\},
\end{equation}
with the Gumbel as reference, and the \emph{residual quantile plot} :
\begin{equation}
\bigg\{\Big(\tilde{z}_{(i)},\ -\log(-\log(i/(m+1)))\Big) \ : \ i=1,\dots,m\bigg\}.
\end{equation}
The choice of the Gumbel as reference distribution can be discussed but this is a reasonable choice regarding its place in EVT.
Figure \ref{fig:ppqqplot2} in \hyperref[app:fig]{Appendix \textbf{\ref{app:fig}}} will present such plots.


\section{Return Levels : New Definitions}\label{sec:returnlvlnstatio}

Whereas we already defined return levels in \hyperref[rlgev]{Section  \textbf{\ref{rlgev}}} for independent sequences, we will now give a more general definition for return levels. 

\subsubsection*{Stationarity} 
Under assumption of a
stationary sequence, the return level is the same for all years. The $m$-year return level $r_m$ is associated with a return period of $m$ years. Let $X_{(n),y}$ denote the annual maximum for a particular year $y$. Assuming $\{X_{(n),y}\}\stackrel{iid}{\sim}F$, there are two main interpretations for return periods in this context, following \citet[chap.4]{ag_extremes_2013} :

\begin{enumerate}
	\item \textbf{Expected waiting time until an exceedance occurs :} let $T$ be the year of the first exceedance. Recalling $F(r_m) = \text{Pr} \big\{X_{(n),y}\leq r_m\big\}=1-1/m$, we write
	
	\begin{equation*}
	\begin{aligned}
	\text{Pr}\{T=t\}=\  & \text{Pr}\{X_{(n),1}\leq r_m,\dots,X_{(n),t-1}\leq r_m,X_{(n),t}>r_m\} \\
	=\ & \text{Pr}\{X_{(n),1}\leq r_m\}\dots\text{Pr}\{X_{(n),t-1}\leq r_m\}\text{Pr}\{X_{(n),t}>r_m\}&  \quad  \boxed{\textnormal{iid assumption}}\\
	= \ & \text{Pr}\{X_{(n),1}\leq r_m\}^{t-1}\ \text{Pr}\{X_{(n),1}>r_m\} &  \quad \boxed{\text{ stationarity}}\\
	=\ & F^{t-1}(r_m)(1-F(r_m))\\
	=\ & (1-1/m)^{t-1}(1/m).
	\end{aligned}
	\end{equation*}
	We easily recognize that $T$ has a geometric distribution with parameter $m^{-1}$. Hence, its expected value is $1/m^{-1}=m$, showing that the expected waiting time for an $m$-year event is $m$ years.
	
	\item \textbf{Expected number of events in a period of \boldsymbol{$m$} years is exactly 1 :} to see that, we define  
	
	\begin{equation*}N=\sum_{y=1}^m \mathbb{I}(X_{(n),y}>r_m)
	\end{equation*}
	as the random variable representing the number of exceedances in $m$ years, with $\mathbb{I}$ the indicator function. Hence, each year is a "trial", and from the fact that \big\{$X_{(n),y}$\big\} are iid, we can compute the probability that the number of exceedances in $m$-years is $k$ by
	\begin{equation*}
	\text{Pr}\{N=k\}=\binom{m}{k}(1/m)^k(1-1/m)^{m-k},
	\end{equation*}
	where we retrieve $N\sim \ \text{Bin}(m,1/m)$ and hence $N$ has an expected value of $m\cdot m^{-1}=1$.
	
\end{enumerate}


\subsubsection*{Non-stationarity}

As demonstrated in \citet[Section 4.2]{ag_extremes_2013}, we can retrieve the same two interpretations of return period as for a stationary process. However, mathematical derivations go beyond the scope of this thesis. Moreover, from definition of non-stationary process, as parameter(s) will be function of time, return levels will also change over time. This will have big impacts on modeling, since an inappropriate model will lead to inappropriate return levels, as we will see to a certain extent in Figure \ref{fig:rl_nsta}. 
We will now try more complex models in order to improve this fit. 


\section{Neural Networks for Nonstationary Series : GEV-CDN}\label{sec:gevcdn}

In the era of Artificial Intelligence and Machine Learning or even the trendy term "Deep Learning", it is interesting to see how artificial Neural Networks (NN) can effectively deal with nonstationarity in EVT for the block-maxima approach. 
In practice, assumptions made by models (\ref{eq:linknsta}) may not be accurate enough, in the sense that it could not take the complex temporal relationship with GEV parameters. One could for example expect to have particular relationships between the covariates\footnote{Here we will still consider the time itself only but we could have other time-varying covariates.} and the GEV parameters. Only considering parametric models in location or scale could thus be seen as too restrictive. For example, \citet{kharin_estimating_2005} allowed for nonlinear trends in temperature extremes by making simulations over a 110-year transient global climate to estimate linear trends in the three GEV parameters based on a series of overlapping 51-year time windows. 

Here, we follow an automated approach allowing to take into account all possible relationships through a flexible modeling approach.
The well-known result from \cite{Hornik_1989} says that provided enough data, hidden units and an appropriate optimization, NN's can capture any smooth dependencies of the parameters given the input, and hence, it can theoretically capture any conditional continuous density, be it asymmetric, multimodal, or heavy-tailed. This can be particularly interesting as we do not have particular prior knowledge on the form of the underlying process of annual maximum temperatures. 
NN's have this facility of being capable of automatically modeling any nonstationary relationships without explicitly specify it a priori, including interactions between covariates.  As demonstrated by the reference article of \citet{cannon_flexible_2010}, physical process such as rainfall or other meteorological data have a tendency to show nonlinearities and so NN's become interesting. However from its flexibility, attention must be given to the danger of overfitting the data.  Another pitfall is its lack of interpretation of the relationships retrieved by the model between inputs and outputs but it bears noting that sensitivity analysis methods as in \citet{cannon_graph_2002} could be used to identify the form of nonlinear relationships between covariates and GEV distribution parameters or quantiles.


We will use a \emph{conditional density estimation network} (CDN), which is a probabilistic variant of the \emph{multilayer perceptron} (MLP). 
An extensive review of the MLP can be found for example in \citet{Hsieh_Tang_1998} in the context of meteorological or climatological predictions. Parameters will be estimated via generalized maximum likelihood. 



\paragraph*{Parametric or nonparametric ?}
It is always a difficult task to state whether Neural Networks (NN) are parametric models or not. NN  are somewhere in the gray area between a \emph{parametric} and a \emph{nonparametric} model, in the sense that it assumes a GEV distribution from the output layer defined by the three parameters of interest, while it also allows for a fabulous flexibility coming from the hidden layers which lead to think that these aree rather nonparametric. 
Note that all transformations applied inside the network are in general parametric and nonlinear.
However, this terminological question is not very relevant and this will not impact the particular modeling.

\subsection{Generalized Maximum Likelihood}


In \hyperref[likgevintro]{Section \textbf{\ref{likgevintro}}} we introduced the concept of penalized likelihood. The idea is that the MLE's may diverge for some values of $\xi$, especially when sample size is small. To resolve this problem, \citet{martins_generalized_2000} suggest the use of a prior distribution for the shape parameter $\xi$ of the GEV model such
that the most probable values of the parameter are included. This method extend the usual ML and is called the \emph{Generalized Maximum Likelihood} (GML). In this method, the penalty is in the form of a prior distribution on $\xi$ : 

\begin{equation}
\pi(\xi)\sim\text{Beta}(\xi+0.5 ; \ c_1,c_2),
\end{equation}
in which $\xi$ is limited to the range $-0.5\leq\xi\leq 0.5$ to limit the search space of $\xi$ during optimization to the support of the shifted beta prior. It is recommended by the authors to set $c_1$ and $c_2$ to 6 and 9 respectively, resulting in a Beta density function with a mode at $-0.1$ and $\approx 90\%$ of the probability concentrated between $-0.3$ and $0.1$. However, these two values can be tuned depending on the application or relying on results of preceding inferential methods, based on the characteristics of the Beta distribution. For a sequence $\boldsymbol{x}=x_1,\ldots,x_n$ of observations, the GML estimator corresponds to the mode of the empirical posterior distribution, i.e. the \emph{generalized-likelihood}

\begin{equation}
\text{GL}(\mu, \sigma,\xi|\boldsymbol{x})=L(\mu,\sigma,\xi|\boldsymbol{x})\cdot \pi(\xi),
\end{equation}
where $L(\cdot)$ can be the log-likelihood $\ell(\cdot)$ (\ref{eq:loglikgevnonstatio}).
When dealing with nonstationnary processes \cite{el_adlouni_generalized_2007} have proven that GML estimators are likely to outperform the usual ML's. This method will be used in the GEV-CDN framework also because it is more flexible and will avoid issues of usual likelihood computations.


\subsection{Architecture of the GEV-CDN Network}



The MLP architecture of the general GEV-CDN framework is pictured in Figure \ref{NN}. Given a set of covariates $\big\{x_i(t), \ i=1,\ldots,I\big\}$ at time $t$, outputs are evaluated following these steps :

\tikzset{%
	neuron missing/.style={
		draw=none, 
		scale=2,
		text height=0.2cm,
		execute at begin node=\color{black}$\vdots$
	},
}
\begin{figure}[!htb]
	\begin{center}
		\resizebox{12cm}{5cm}{ \fbox{
				\begin{tikzpicture}[x=1.5cm, y=1.5cm, >=stealth]
				\tikzstyle{annot} = [text width=4em, text centered]
				
				\foreach \m [count=\y] in {1}
				\node [circle,fill=green!50,minimum size=.7cm ] (input-\m) at (0,1) {};
			
				\foreach \m [count=\y] in {2}
				\node [circle,fill=green!50,minimum size=.7cm ] (input-\m) at (0,0.35) {};
				
				\foreach \m [count=\y] in {3}
				\node [circle,fill=green!50,minimum size=.7cm ] (input-\m) at (0,-0.75) {};
				
				\node [neuron missing]  at (0,-0.25) {};
				
				\foreach \m [count=\y] in {1}
				\node [circle,fill=red!50,minimum size=.7cm ] (hidden-\m) at (2,0.75) {};
				
				\foreach \m [count=\y] in {2}
				\node [circle,fill=red!50,minimum size=.7cm ] (hidden-\m) at (2,-0.38) {};
				
				\node [neuron missing]  at (2,0.1) {};
				
				\foreach \m [count=\y] in {1}
				\node [circle,fill=blue!50,minimum size=.7cm ] (output-\m) at (4,1.8-\y) {};
				
				\foreach \m [count=\y] in {2}
				\node [circle,fill=blue!50,minimum size=.7cm ] (output-\m) at (4,1.05-\y) {};
				
				\foreach \m [count=\y] in {3}
				\node [circle,fill=blue!50,minimum size=.7cm ] (output-\m) at (4,0.35-\y) {};
				
				\draw [<-] (input-1) -- ++(-1,0)
				node [above, midway] {$x_1(t)$};
				\draw [<-] (input-2) -- ++(-1,0)
				node [above, midway] {$x_2(t)$};
				\draw [<-] (input-3) -- ++(-1,0)
				node [above, midway] {$x_I(t)$};
				
				\node [below] at (hidden-1.north) {$h_1$};
				\node [below] at (hidden-2.north) {$h_J$};
				
				\draw [->] (output-1) -- ++(2.4,0)
				node [above, midway] {$\boldsymbol{\mu(t)=o_1(t)}$};
				\draw [->] (output-2) -- ++(2.4,0)
				node [above, midway] {$\sigma(t)=\exp(o_2(t))$};
				\draw [->] (output-3) -- ++(2.4,0)
				node [above, midway] {$\xi(t)=\xi^*\tanh(o_3(t))$};
				
				\foreach \i in {1,...,3}
				\foreach \j in {1,...,2}
				\draw [->] (input-\i) -- (hidden-\j);
				
				\foreach \i in {1,...,2}
				\foreach \j in {1,...,3}
				\draw [->] (hidden-\i) -- (output-\j);
				
				\node[annot,above of=hidden-1, node distance=1cm] (hl) {hidden layer};
				\node[annot,above of=input-1] (il) {input layer};
				\node[annot,above of=output-1] {output layer};
				\node[annot, above =2.6cm, right=0.7cm, node distance=1cm]{\textcolor{red}{\textbf{(1)}}};
				\node[annot, above =1.2cm, right=3.7cm]{\textcolor{blue}{\textbf{(2)}}};
				\end{tikzpicture}
			}
		}
		\caption{General framework of the fully-connected nonstationary GEV-CDN based on \citet{cannon_flexible_2010}. The input layer will still the time itself in our application, i.e. $x_i(t)=t,  \ \forall i =1,\ldots,I$ but it can be other covariates. The hidden layer represent additional complexity incorporated in the model and the output layer represent the three GEV parameters. \textbf{\textcolor{red}{(1)}} and \textbf{\textcolor{blue}{(2)}} represent the functional relationships (\ref{eq:nnlay1})-(\ref{eq:nnlay2}) between layers. }\label{NN}
	\end{center}
\end{figure}


\begin{itemize} 

\item The $j$-th hidden layer node $h_j$ is given by 
\begin{equation}\label{eq:nnlay1}
h_j(t)=m\Big(\sum_{i=1}^{I}x_i(t)\cdot  w_{ji}^{\textbf{\textcolor{red}{(1)}}}+b_j^{\textbf{\textcolor{red}{(1)}}}\Big),
\end{equation}
with $m(\cdot)$ the hidden layer activation function, $w_{ji}^{(1)}$ and $b_j^{(1)}$ are the input-hidden layer weight and bias. The function $m(\cdot)$ is often sigmoidal to allow the GEV-CDN mapping to be nonlinear but it can be the identity function for a strictly linear mapping.

\item The value of the $k$-th output is given by 
\begin{equation}\label{eq:nnlay2}
o_k(t)=\sum_j^Jh_j(t)\cdot w_{kj}^{\textbf{\textbf{\textcolor{blue}{(2)}}}}+b^{\textbf{\textbf{\textcolor{blue}{(2)}}}}_k, \qquad k=1,2,3 ,
\end{equation}
where $w_{kj}^{(2)}$ is the hidden-ouput layer weight and $b^{(2)}_k$ is the hidden-output layer bias.
	
\item The GEV parameters are obtained by applying the output-layer activation functions $g_k(\cdot)$ denoted in Figure \ref{NN}. As usual, the function $g_2(\cdot)$ is to force $\sigma$ to take positive values and $g_3(\cdot)$ is to constraint $\xi$ to lie within $[-\xi^{*},\ \xi^{*}]$. Again, we notice that it is not recommended to allow $\xi$ to vary with time. 
	
\end{itemize}

A hierarchy of models can be defined by varying the structure of the CDN (number of hidden layers, which activation function $m(\cdot)$ and weights connections)  and compared by the selection criterion such as the BIC or $\text{AIC}_{\text{c}}$ discussed in (\ref{aicbic}). 


\subsection{Prevent Overfitting : Bagging} 


\emph{B}ootstrap \emph{agg}regat\emph{ing} or \emph{bagging} is an ensemble method\footnote{For climatologists, \emph{ensemble models} are different but of major utility, especially to make weather forecasting, see for example \citet{suh_development_2012} among others.} used in many state-of-the-art machine learning algorithms such as Random Forests discovered by \citet{Breiman_2001}. This technique is praised in Machine Learning for its performance as it decreases variance of predictions (or estimates) and hence reduce the risk of overfitting. This method works by generating additional data with repetitions from the original dataset to produce multisets of the same size.
The  individual  multisets’  outputs  having equal  weights  are  then  combined  by  averaging the ensemble members. This process is also known as \emph{model averaging}.
Indeed, by increasing the size of original data you cannot improve the model predictive force, but just decrease its variance.

 
\citet{Carney_2005} have successfully applied this averaging method in the context of CDN and \citet{cannon_flexible_2010} says it is worth exploring for GEV-CDN models. He implemented it soon after in its R package \texttt{GEVcdn}. \emph{Early stopping} stopping can be added as a computationally intensive means of controlling overfitting as it stops training prior to convergence of the optimization algorithm and hence allow to reduce the complexity of the model.

Other techniques are available to prevent overfitting. First used by \citet{mackay92b}, \emph{weight penalty regularization} is popular and available in the GEV-CDN framework as a means of limiting the effective number of parameters. The amount of weight penalty is
controlled via a Gaussian prior on the magnitude of the input-hidden layer weights. Optimal value for the
variance of the Gaussian prior must be set relying on some form of split-sample or cross validation scheme.



\subsection{Confidence Intervals : Bootstrapping Methods}

Like the estimated parameters themselves, the standard errors may not be reliable for small samples. One way to improve the accuracy of the standard errors is to use \emph{bootstrap}. Discovered by \citet{efron1979}, this can be used for EV dfs including nonstationarity. The bootstrap samples are manufactured through
Monte Carlo resampling of residuals to attend to the underlying assumption  that original sample is iid.
There exist a large panel of different bootstrap procedures to construct confidence intervals.  For example, \citet{2006JHyd..329..534K} follows these steps for the residual bootstrap : 
 
\begin{enumerate}
	\item\label{itboot} Fit a nonstationary GEV model to the data.
	\item Transform residuals from the fitted model to be identically distributed :
	\begin{equation*}
	\varepsilon(t)=\bigg[1+\xi(t)\cdot\sigma^{-1}(t)\cdot\Big(x(t)-\mu(t)\Big)\bigg]^{-\xi^{-1}(t)},
	\end{equation*}
	where $\varepsilon(t)$ is the $t$-th trandformed residual.
	
	\item Resample $\varepsilon(t)$ with replacement to form the bootstrapped set $\Big\{\varepsilon^{(b)}(t), \ t=1,\ldots,n\Big\}$.
	
	\item Rescale the bootstrapped residuals by inverting the transformation : 
	
	\begin{equation}\label{eq:boots}
	x^{(b)}(t)=\mu(t)-\sigma(t)\cdot\xi^{-1}(t)\cdot\Big(\varepsilon^{(b)}(t)-1\Big).
	\end{equation}
	\item\label{it5:boot} Fit a new nonstationary GEV  model to the boostrapped samples (\ref{eq:boots}) and estimate the parameters and quantiles from the fitted model. 
	\item Repeat steps \ref{itboot} to \ref{it5:boot} a large number of times B.
	
\end{enumerate}

\citet{cannon_flexible_2010} found that \emph{parametric} bootstrap outperformed the \emph{residual} bootstrap in the GEV-CDN framework, but he did not consider alternative bootstrap approaches such as the \emph{bias-adjusted percentile} which might yield better calibrated confidence intervals.
Empirical Monte-Carlo comparisons of coverage from all available methods considered so far would be interesting. Furthermore, the incoming chapter will introduce other methods to construct intervals in a strictly Bayesian approach  relying also on Monte-Carlo sampling.
