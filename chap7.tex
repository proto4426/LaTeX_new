\addcontentsline{toc}{subsection}{Note on the available R package : \texttt{evdbayes}}
\subsection*{Note on the available R package : \texttt{evdbayes}}

The \texttt{evdbayes} is the only package available on CRAN for (see \citet{ribatet_users_2006}) which is a very old package. We had problems to understand both its structure and the famous "black-box"
It only use the Metropolis Hastings algorithm.

 Note that efforts should put to the new \texttt{revdbayes} R package which use the ratio-of-uniforms method is an other acceptance-rejection type of simulation algorithm. ( \hyperref[sec:bayes_ratio]{Section \textbf{\ref{sec:bayes_ratio}}}
 
\section{From Our Functions (R package)}



\section{From HMC algorithm using STAN language}

The problem is maybe from \ref{likgevintro}. The parameter $\xi$ is relatively near the region that could be problematic, causing convergence issues. 



\section{Ratio of Uniform : \texttt{revdbayes} package}\label{sec:bayes_ratio}

\url{https://cran.r-project.org/web/packages/revdbayes/vignettes/revdbayes-vignette.html}



From the \texttt{revdbayes}

Helped with \texttt{rust}
\section{Comparisons}

\citet{hartmann_bayesian_2016}
We can calculate the effective sample size (ESS) using the posterior samples for each parameter : 

\begin{equation}
\text{ESS}=N\cdot (1+2\sum_k\gamma(k))^{-1}
\end{equation}

where $N$ is still the number of posterior samples and $\gamma(k)$ are the monotone lag $k$ sample autocorrelations. \citet{geyer__1992}. We can thus interpret this as the number of effectively independent samples. 


\subsection{STAN}

Benefits : 

\begin{itemize}
	\item Allows more flexibility (?) through the mathematical formulation of the formula
	\item It is really smoother and clearer (straightforward) for this kind of problems 
\end{itemize}

Drawbacks : 

\begin{itemize}
	\item New language with all the problems/errors arising when learning it. 
\end{itemize}



\section{Comparison with frequentists results}

In this first analysis, we rely on 
