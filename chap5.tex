\begin{definition}[Posterior distribution]
	Let $\boldsymbol{x}=(x_1,\dots,x_m)$
	denote the observed data of a random variable $X$ distributed according to a distribution with density function 
	
	
	\begin{equation}\label{bayeseq}
	\pi (\theta|\boldsymbol{x})=\frac{\pi(\theta)\cdot L(\theta|\boldsymbol{x})}{\int_{\Theta} 
		\pi(\theta)\cdot L(\theta|\boldsymbol{x}) \cdot d\theta}\propto \pi(\theta)\cdot 
	L(\theta|\boldsymbol{x})
	\end{equation}
	where $L(\cdot)$ denotes the likelihood function, as in \ref{} but there it was the \underline{log}-likelhiood !!!
	and $\theta$ usually denotes the multidimensional set of parameters in EVT, $\theta=(\mu,\sigma,\xi)$, at least in a univariate stationary context.
\end{definition}

\begin{enumerate}
	
	\item Whenever it is possible, it allows to introduce to introduce other source of knowledge coming from the domain at-hand, by the elicitation of a prior.  The counter-argument of this advantage is that it also introduces (improper ?) subjectiveness.
	\item\label{it2bayes} "account-
	ing for parameter and threshold uncertainty is perhaps handled most easily in the
	Bayesian paradigm" \cite[pp.106]{dey_extreme_2016}
	
	As such, It permits an elegant way of making future predictions  which is one of the most(?) important issue in EVT.
	
	\item Bayesian framework can overcome the regularity conditions of the likelihood inference
	(see \hyperref[likintro]{section \textbf{4.1}}).
	Thus it usually provides a viable alternative in cases when MLE (for example) breaks down. And actually, we are not so fo far from the problematic situations depicted in section \textbf{3.1}. 
	Moreover, the Highest Posterior Probability (HPD) region is constructed so that... and there is no more need to fall to asymptotic theory as in conventional methods.
	
	
	\item For an asymmetric distribution,
	the HPD interval can be a more reasonable summary than the central probability
	interval ( see illustration ...). For symmetric densities, HPD and central interals are the same while HPD is shorter for asymmetric densities.
	See \citet{liu_simulation-efficient_2015}....
	
\end{enumerate}



As the dependence becomes stronger, the run length n must be larger in order
to achieve the same precision. Dependence exists both within the output for a single parameter
(autocorrelations) and across parameters (cross-correlations), we discuss this issue in section \hyperref[label]{text}.




\section{Prior Elicitation}

Sometimes viewed as advantage from the amount of information that can be retrieved, and sometimes viewed as an drawback due to the (rather unquantifiable) subjectivity that introduced, the construction of the prior is a key step in Bayesian analysis.


Priors are necessary in the Bayesian paradigm to be able to compute the posterior in  (\ref{bayeseq}).But, priors require the legitimate statement of domain's expert, to make this viewed the less subjective as possible

Prior may not be of great importance if the size ($m$) of the dataset is large. It can be seen from (\ref{bayeseq}) where the amount of information contained in the data through $L(\theta|\boldsymbol{x})$ will be prominent compared to this contained in the prior through $\pi(\theta)$. Prior will have limited influence.

One is aware that this is not often the case in EVT cases. By design, we are dealing with rather small so-constructed datasets. And mostly for this reason, it could be important to incorporate additional information in this limited dataset through the prior distribution.

\subsection{Non-informative Priors}

Receive a correct and accepted advice from an expert is often difficult
So, in many cases, we cannot inject information through the prior. 
We must then construct a prior which represent this lack of knowledge so that  they do not influence posterior inferences. 

There exists a vast amount of uninformative priors in the literature (see e.g. \cite{yang_catalog_1996}, \cite{ni_noninformative_2003})
This family of priors can be \emph{improper}, i.e. priors for which the integral of $\pi(\theta)$ over the parameter space is not finite. It is valid to use improper priors only if the posterior target is proper. 

Adjustments of these priors must always be thought in practical applications
\subsubsection*{Jeffrey's prior} 
is specified as 

\begin{equation}
\pi(\theta)\propto \sqrt{\det I(\theta)}, \qquad \text{where }\quad\quad I_{ij}(\theta)=\mathbb{E}_{\theta}\Bigg[-\frac{\partial^2\log f(\boldsymbol{x}|\theta)}{\partial\theta_i\partial\theta_j}\Bigg], \ \ i,j= 1,\dots,d.
\end{equation}
where $f(\boldsymbol{x}|\theta)$ is of course the density function of X.

This prior is invariant to reparametrization, but has complex form for EV models, and it exists only when $\xi>-0.5$ in GEV models, where it is function of $\xi$ and $\sigma$ only.


\subsubsection*{MDI prior}
Maximal Data Information priors


However, it has been showed by \cite{northrop_posterior_2016} that 
both Jeffrey and MDI priors give improper posterior when there are no truncation of the shape parameter, i.e. we must restrict the fact that $\pi(\theta)\rightarrow\infty$ as $\xi\rightarrow(-)\infty$ for Jeffreys (MDI), in order to obtain a proper posterior.

\subsubsection*{Vague priors}
The last and often preferred alternative to construct uninformative priors is to use proper priors which are near flat, e.g. which are uniform or with exhibits large variance for the normal distribution.

In GEV we will take independent normal-distributed priors each with a large (tuned) variance. When these variances increase, we get at the limit

\begin{equation}
\pi(\theta)=\pi(\mu,\nu,\xi)\stackrel{(\independent)}{=}\pi(\mu)\cdot\pi(\nu)\cdot\pi(\xi)\ \propto 1,
\end{equation}

where $\nu= \log\sigma$.

Taking multivariate normal distribution as prior has also been proposed (see ) is often difficult as it involves 9 (hyper)parameters in total and this can be difficult 


\subsection{Informative Priors}

STAN : "It can
also be a huge help with computation to have less diffuse priors, even
if they're not informative enough to have a noticeable impact on the posterior. "

\paragraph*{Gamma Distributions for Quantile Differences}


\paragraph*{Beta Distributions for Probability Ratios}



\subsubsection*{The Bayes Factor}




\section{Bayesian Computation : Markov Chains}


Methods have been developed for sampling from arbitrary posterior distributions $\pi(\theta|\boldsymbol{x})$. Simulations of $N$ values $\theta_1,\theta_2,\dots,\theta_N$ that are iid from $\pi(\theta|\boldsymbol{x})$ can be used to estimate features of interest.

But simulating from $\pi(\theta|\boldsymbol{x})$ is usually not achievable and this is why we need \textbf{Markov Chain Monte Carlo} (MCMC) techniques. 
We use it to simulate a markov chain $\theta_1,\theta_2,\dots,\theta_N$ that conerge to the target distribution $\pi(\theta|\boldsymbol{x})$.
This means that, after some \emph{burn-in period} $B$, $\theta_{B+1},\dots,\theta_N$ can be treated as random sample from $\pi(\theta|\boldsymbol{x})$.


Let's now (a bit weakly) define one of the most important results in Markov Chain theory.

\begin{definition}[\emph{First-order discrete-time} Markov Property]
	Let $k_0,k_1,\dots$ be the states associated to a sequence of time-homogeneous random variables, say $\big\{\theta_t:t\in\mathbb{N}\big\}$.
	The Markov property states that the distribution of the future state $\theta_{t+1}$ depends only on the distribution of the current state $\theta_{t}$. 
	In other words, given $\theta_{t}$, we have that $\theta_{t+1}$ is independent of all the states prior to $t$. We can write this as
	
	\begin{equation}
	\text{\emph{Pr}}\big\{\theta_{t+1}=k_{t+1}\ |\ \theta_t=k_t,\ \theta_{t-1}=k_{t-1},\dots\big\} = \text{\emph{Pr}}\big\{\theta_{t+1}=k_{t+1}\ | \ \theta_t=k_t\big\}.
	\end{equation}
\end{definition} 

or see \citet[section 2.2.3]{angelino_patterns_2016} for more in-depth results.

The samples are not independent, and the dependence influences the accuracy of the posterior estimates. As dependence becomes stronger, we must increase the run-length $N$ to achieve the same accuracy. 


\subsection{Algorithms} 

We are looking for a so-generated chain that has a \underline{stationary} distribution $\pi(\theta|\boldsymbol{x})$. This is the case if the chain is 

\begin{enumerate}
	\item \emph{aperiodic}
	\item \emph{irreducible} or \emph{ergodic}, that is if any state for $\theta$ can be reached with probability $>0$ in a finite number of steps from any other state for $\theta$.
\end{enumerate}

"The Markov chains Stan and other MCMC samplers generate are \emph{ergodic} in the
sense required by the Markov chain central limit theorem, meaning roughly that there
is a reasonable chance of reaching one value of theta from another." \cite{stan_stan_2016}

With MH or Gibbs sampler, we need to tune individually the proposal standard deviations to reach a correct acceptance, and this is often done with trial-and-error methodology. 



The  performance  of  the standard  Markov  chain  Monte  Carlo  estimators  depends  on  how  effectively the Markov transition guides the Markov chain along the neighborhoods of high probability.  If  the  exploration  is  slow  then  the  estimators  will  become  computationally inefficient,  and  if  the  exploration  is  incomplete  then  the  estimators  will  become  biased
\citet{betancourt_diagnosing_2016}. It is then necessary to consider other form of sampling...


\subsection{Hamiltonian Monte Carlo}

Package Rstan

\cite{neal_mcmc_2011} and \cite{betancourt_hamiltonian_2015} are really 


HMC permit to better exploit the properties of
the target distribution to make informed jumps through neighborhoods of high probability while avoiding neighborhoods of low probability entirely.


\subsection{Computational efficiency comparison}

In modern statistical area, computing methods have been widely ... 
And this need for compiutations will rise in the future. 

We will then compare our 3 methods too see if effectively


\section{Convergence Diagnostics}

When applying MCMC algorithms to estimate posterior distributions, it is vital to assess convergence of the algorithm to try to ensure that we reached the stationary target distribution. Let's now enumerate some of the key steps we must keep in mind when thinking about convergence, an hence reliable results.

\begin{enumerate}
	\item A sufficient \emph{burn-in period} $B<N$ must be chosen to ensure that the convergence to the posterior distribution $\pi(\theta|\boldsymbol{x})$ has occurred. 
	\item For the same reason, a sufficient number of simulations $N$ to eliminate the influence of initial conditions and ensure accuracy in the estimations ((and then make sure than we are sampling from the target stationary (posterior) distribution)).
	\item Several dispersed starting values must have been simulated to ensure we explored all the regions of high probability. This is particularly important when the target distribution is complex.
	\item\label{convdiag4} The chains must have good mixing properties, in the sense that the whole parameter space (...) 
	A common technique that we will apply is to run different chains several times and then combine a proportion of each chain (typically $50\%$) to get the final chain. This procedure wants to ensure a proper mixing behaviour. 
	The potential scale reduction factor (Gelman diagnostic) is also a popular tool, see .
\end{enumerate}

We must keep in mind that no convergence diagnostics can prove that convergence really happened and validate the "model".
However, a combined use of several relevant diagnostics will be required to increase our confidence that convergence actually happened.

\subsection{Proposal Distribution}



The main ideas are : 

\begin{itemize}
	\item If the variance of the proposal distribution is too large, most proposals will be rejected : 
	
	 ie the jumps through the chain are too large,
	\item If the variance of the proposal distribution is too low, then most proposals will be accepted
\end{itemize}

Both are harmful for the objective of an efficient "visit" of the whole parameter space. 


Widely speaking, we consider 2 different  types of algorithms in which it is preferable to target a certain acceptance rate. It is distinguished by the updating manner of the components of $\theta$ through the algorithm, i.e. the 3 univariate parameters of interest.


\begin{itemize}
	\item When all components of $\theta$ are updated simultaneously, it is recommended to target an acceptance rate of around 0.20.
	 \citet{Roberts_weak_1997} have shown that, under quite general conditions, the asymptotically optimal acceptance rate is 0.234. (for  target density that has a symmetric product form ) 
	This quantity has been verified by \citet{Sherlock_optimal_2009}. It holds for the \emph{Metropolis-Hastings} algorithm.
	
	\item When the components are updated one at a time, an acceptance rate of around 0.40 is recommended. It holds for the\emph{ Gibbs sampler} algorithm. 
\end{itemize}

Let's  (see \citet{Bédard_optimal_2008} for example for the first case)



\subsubsection*{Gelman-Rubin diagnostic : the $\hat{R}$ statistic}

As discussed in \hyperref[convdiag4]{item \textbf{4}} above

\subsubsection*{Geweke diagnostic}



\subsubsection*{Thinning}

iteration $k$ is stored only if $k \mod$ thin is zero (and if $k$ greater than or equal to the burn-in B).

This typically reduces the precision of posterior estimates, but it may represent a necessary computational saving.


\subsection{The problem of auto and cross-correlations in the chains}


There exists exists 2 problems of correlations in the output delivered by a MC. 

\begin{itemize}
	\item \textbf{Autocorrelation} is the
	\item \textbf{Cross-correlation}
\end{itemize}


\section{Posterior Predictive}
notation for posterior ? pi or f


As discussed in \hyperref[it2bayes]{item \textbf{2}} above, prediction is of important interest in EVT, and this is "facilitated" in the Bayesian paradigm. This also permits a more straightforward quantification of the inferential uncertainty associated.

\begin{definition}[Posterior Predictive density]
	Let $X_{m+1}$ denotes a (one-step-ahead) future observation with density $f(x_{m+1}|\boldsymbol{\theta})$. Then we define the Posterior Predictive density of a future observation $X_{m+1}$ given $\boldsymbol{x}$ as 
	
	\begin{equation}
	\begin{aligned}
	f(x_{m+1}|\boldsymbol{x})
	= & \int_{\Theta}f(x_{m+1},\theta | \boldsymbol{x})\cdot d\theta=\int_{\Theta} f(x_{m+1}|\theta)\cdot \pi (\theta|\boldsymbol{x})\cdot d\theta
	\\ := & \ \ \mathbb{E}_{\theta|\boldsymbol{x}}\big[f(x_{m+1}|\theta)\big]
	\end{aligned}
	\end{equation}
	where the last line emphasizes that we can evaluate $f(x_{m+1}|\boldsymbol{x})$ by averaging over the different possible parameter values.
	
\end{definition}

The uncertainty in the model is reflected here through $\pi(\theta|\boldsymbol{x})$ while the uncertainty due to variability in future observations is also reflected through $f(x_{m+1}|\theta)$.

\begin{definition}[Posterior Predictive probability]
	The posterior predictive probability of $X_{m+1}$ exceeding some threshold $x$ is accordingly given by
	\begin{equation}
	\begin{aligned}
	\text{\emph{Pr}}\{X_{m+1}>x\ | \ \boldsymbol{x}\}= &\int_{\Theta}\text{\emph{Pr}}\{X_{m+1}>x \ | \ \theta \} \cdot \pi(\theta|\boldsymbol{x})\cdot d\theta \\ 
	= & \ \mathbb{E}_{\theta|\boldsymbol{x}}\big[\text{\emph{Pr}}(X_{m+1}>x \ | \ \theta)\big]
	\end{aligned}
	\end{equation}
\end{definition}

This quantity is often of interest in EVT as we are rather concerned with the probability of future unknown observable exceeding some threshold.

However, this quantity is difficult to obtain analytically. Hence, we will more rely on simulated approximations. Given a sample $\theta_1,\dots,\theta_r$ from the posterior $\pi(\theta|\boldsymbol{x})$, we use 

\begin{equation}
\text{Pr}\{X_{m+1}>x\ | \ \boldsymbol{x}\}\approx r^{-1}\sum_{i=1}^r\text{Pr}\{X_{m+1}>x \ | \ \theta_i\},
\end{equation}

where $\text{Pr}\{X_{m+1}>x \ | \ \theta_i\}$ follows directly from $f(x|\theta)$. 

We will now analyse more in-depth the numerical computations in the Bayesian paradigm or how we can get numerically a sample of the posterior distribution.



\section{Bayesian Predictive Accuracy for Model Validation}

\subsection{Cross-validation for predictive accuracy}

When having large amount of data, we can use a well-known and widely used technique coming from Machine Learning. That is, dividing the dataset between a training (typically $75\%$ of the whole set) and a test set containing the remaining observations.
For example, having $N$ draws $\theta^{(1)},\theta^{(2)},\dots,\theta^{(N)}$ coming from the posterior $\pi(\theta|x_{train})$, we can score each value using (?)


\begin{equation}
\log\bigg[N^{-1}\sum_{t(i)?=1}^N f(x^*|\theta^{(t)})\bigg].
\end{equation}

However, we often do not have large amounts of data. Henceforth, we can use the \emph{cross-validation} technique which is more relevant in smaller dataset, but which is computationally more demanding. There exists several variants of them. 

\subsubsection*{Leave-one-out cross-validation}

The \emph{Leave-One-Out} (LOO) cross-validation is the 


\subsubsection*{$K$-fold cross-validation}

\citet{Vehtari_practical_2016}


Or we can use other criteria which avoid the computations. The basic approach is to use  $\log f(x|\bar{\theta})-p^*$ for $N$ draws $\theta^{(1)},\dots,\theta^{(N)}$ from $\pi(\theta|x)$. where $p^*$ represents the effective number of parameters and $\bar{\theta}$ the posterior mean. Several methods exists using this idea. We will see the two most important.

\subsubsection*{Deviance Information Criterion}

The \emph{Deviance Information Criterion}
(DIC) was first used by \citet{Spiegelhalter_bayesian_2002} and use the following estimate for the effective number of parameters 

\begin{equation}
p^*=2\cdot \bigg(\log f(x|\bar{\theta})-N^{-1}\sum_{t=1}^N\log f(x|\theta^{(t)})\bigg)
\end{equation}

It is defined on the deviance scale and smaller DIC values indicate better models.

\begin{equation}
\text{DIC}= 2\log f(x|\bar{\theta}) - \frac{4}{N} \sum_{t=1}^N\log f(x|\theta{(t)})
\end{equation}



\subsubsection*{Widely Applicable Information Criterion}

The \emph{Widely Applicable Information Criterion} (WAIC) is a more recent approach proposed by \citet{Watanabe_asymptotic_2010} and is given by 


\begin{equation}
\text{WAIC}= 2 \sum_{i=1}^n\big[\log(\mathbb{E}_{\theta|x}f(x_i|\theta)\big]- \mathbb{E}_{\theta|x}\log f(x_i|\theta) 
\end{equation}

or 

\begin{equation}
\text{WAIC} = \sum_{i=1}^n\Bigg[2\log \bigg(N^{-1}\sum_{t=1}^N f(x_i|\theta^{(t)})\bigg)-\frac{4}{N}\sum_{t=1}^N\log f(x_i|\theta^{(t)})\Bigg]
\end{equation}



There exists for sure other several methods, as proposed by \citet{gelman_understanding_2014}.
 
 'LOO and WAIC have various advantages over simpler estimates of predictive error such asAIC and DIC but are less used in practice because they involve additional computational steps' 


For each generated chains with dispersed starting values, we evaluate separately the information criteria. The discrepancies between the chains are small (?), which is a good sign. 

\section{Bayesien Inference ?}

\subsection{Bayesian Credible Intervals}\label{bayes_cred_int}

The Bayesian \emph{credible intervals} are inherently different from the frequentist's confidence intervals. In the Bayesian intervals, the bounds are treated as fixed and the estimated parameter as a random variable, while in the frequentist's setting, bound are random variables and the parameter is a fixed value.

There exist mainly two kinds of credible interval in the Bayesian sphere : 

\begin{itemize}
	\item The \emph{Highest Probability Interval} (HPD) which is defined as the shortest interval containing $x\%$ of the posterior probability, e.g. if we want a $95\%$ HPD interval $(\xi_0,\xi_1)$ for $\xi$ :
	
	\begin{equation} 
	\int_{\xi_0}^{\xi_1}\pi(\xi|\boldsymbol{x})d\xi=0.95 \qquad\text{with}\qquad \pi(\xi_0|\boldsymbol{x})=\pi(\xi_1|\boldsymbol{x}).
	\end{equation}
	
	It is often the preferred interval as it gives the parameter's values having the highest posterior probability. 
	\item The Quantile-based credible intervals or \emph{equal-tailed interval} picks an interval which ensures a probability of being below this interval as likely as of being above it. 
	For some posterior distribution which are not symmetric, this could be misleading, thus it is not the most recommended interval. (see ..)
	However, these are often easily obtained when we have a random smaple of the posterior...(?)
\end{itemize}


\subsection{Distribution of Quantiles : Return Levels}

The Markov chains generated can be transformed to estimate quantities of interest such as quantiles and hence return levels.

The values can be retrieved in the same manner as we have done in the GEV frequentist setting in (\ref{rleqgev}). If the df $F$ associated is GEV then $y_m = -\log(1-m^{-1})$, and if $F$ is GPD then $y_m=m^{-1}$.

$r_m$ is the quantile corresponding to the upper tail probability $p=m^{-1}$. 

We can use the values of the samples generated by the posterior to estimate features of this distribution. (... see edbayes)
 
 

\section{Bayesian Model Averaging}
