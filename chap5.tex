\addcontentsline{toc}{section}{Repository for the code : R Package}
\section*{Repository for the code : R Package}

The created \textbf{R package} can be easily downloaded from this \href{https://github.com/proto4426/PissoortThesis}{\textbf{repository}} :
\begin{center}\label{xxx}
 \url{https://github.com/proto4426/PissoortThesis}
\end{center}
by following instructions in the README. It follows the standard structure of a usual R package (see e.g. \citet{leisch_creating_2008}) and makes use of the \texttt{roxygen2} package for the documentation.

For your convenience, an external folder has been created in this repository containing all the scripts created during this thesis. It allows to reproduce all the results often with more details, tables or plots. It is located in the \textbf{/Scripts-R/} folder of the repository. Each script will be mentionned in its corresponding section. Further details on the repository structure can be found in \hyperref[appgit]{Appendix \textbf{\ref{appgit}}}.

\addcontentsline{toc}{subsection}{Visualization Tool : Shiny Application}
\subsection*{Visualization Tool : Shiny Application}

Shiny applications that have been developed can be run directly through the R environment after loading the package in your environment, by executing 

\begin{center}
\begin{lstlisting}[language=R]
runExample()  # in the R console,
\end{lstlisting}
\end{center}
choosing one applications's name between the displayed propositions and write it inside (' ').
So far, we have
\begin{itemize}
	\item[-]('GEV\_distributions') :  application based on Figure \ref{gevdens} smoothly displaying the GEV and the influence of its parameters.
	
	\item[-] ('trend\_models') :
	application dedicated to annual maxima that can you can visualize with some preliminary methods (see \hyperref[sec:firstvisu]{Section \textbf{\ref{sec:firstvisu}}}).
	
	\item[-] ('splines\_draws') : application to simulate splines of GAM model that allow to visualize coverages of the pointwise and simultaneous confidence intervals. (see \hyperref[sec:splines]{Section \textbf{\ref{sec:splines}}}).

	\item[-] ('neural\_networks') : 
	application for the \texttt{GEVcdn} allowing bagging to see the effects of all the (hyper)parameters, since the model's choice is still subjective... (see if we have time to finish !)
	
\end{itemize}
Moreover, a dynamic overview of the applications is available in the repository's \texttt{README}.

\paragraph*{All-in-one : Dashboard} For the reader's convenience, we have decided to gather all applications in a smooth \emph{dahsboard}. Moreover, it is uploaded on a server at the following \href{https://proto4426.shinyapps.io/All\_dashboard/}{URL}\footnote{\url{https://proto4426.shinyapps.io/All\_dashboard/}}
 in order to provide a quick user-friendly overview through the \texttt{rsconnect} interface.
 
The rest of the analysis in this chapter will rely on \texttt{1intro\_stationary.R} and \\ \texttt{1intro\_trends(splines).R} codes from the \textbf{/Scripts-R/} folder of the \href{https://github.com/proto4426/PissoortThesis}{repository}.


\section{Presentation of the Analysis : Temperatures from Uccle}\label{sec:presuccle}

Data used during this thesis comes from the "Institut Royal de Météorologie" (IRM) located in Uccle. We were provided official data used by meteorologists in Belgium. We wanted to have data that is reliable enough in order to be able to produce relevant results. 

We were provided a set of databases from the IRM and decided to focus on temperature analysis in order to assess climate warming. Rainfall data analysis would have also been interesting in EVT. 

The fact that the dataset begins only at year $1901$ is due to homogeneity reasons since the measurement shelters evolved substantially compared to previous periods, especially in terms of measurement errors. 
For meteorological considerations and again for reasons of homogeneity, it is better to analyze the temperatures in \emph{closed shelters}, following for example \citet{lindsey_use_1956}. Indeed, thanks to helpful advice from C.Tricot, climatologist at the IRM, temperatures in closed shelters have the advantage of not being too influenced by the solar radiation which artificially increases temperatures, especially in periods of \emph{extreme} heats and thus high solar activity, and therefore in the studied case. 
For example, the maximum temperature of $36.6^{\circ}c$ that occurred the 27th June 1947 in closed shelters was actually measured at $38.8^{\circ}c$ in open shelters the same day. 



\subsection*{Comparisons with freely available data} 

A similar dataset for Uccle is publicly available on the Internet\footnote{\url{http://lstat.kuleuven.be/Wiley/Data/ecad00045TX.txt}}. It was a project initially performed by the KMNI and which was used by \citet{beirlant_statistics_2006}. However, we did not want to simply analyze that data as we did not know if it was trustworthy, for reasons mentioned above.

Afterwards, we compared these two datasets for years $1901$ to $2001$. We remark that there are large differences
in these two datasets. For example,
$54\%$ of measurements are equal with those of the dataset in open shelters against $14.4\%$ in the closed shelters. For this reason, we have confidence that the public dataset has temperatures measured in open shelters, which is not recommended. Hence, we conclude that large measurement errors can easily occur in unofficial data which emphasizes the necessity of reliable data in order to yield a reliable analysis.

\section{First Analysis : Annual Maxima}\label{sec:firstana}


\subsection{Descriptive Analysis}

The goal of this section is simply to display a global overview of the data. The code provided most of the descriptive analysis. To summarize most of the information, we have plotted a violin-plot and a density plot of the \textbf{daily} maximum temperatures in Figure \ref{fig:violin_density} in \hyperref[app:fig]{Appendix \textbf{\ref{app:fig}}} where we divided data in each meteorological season to point out the differences. Per example, we can see that the the spread is higher for autumn and spring.

Now, we will rather stick to the block-maxima approach by taking yearly blocks leaving us with $n=116$ data which seems justifiable for further GEV analysis and the conditions to hold. We will discuss the choice of the block length further in the \hyperref[sec:anagev]{next Chapter}.

\subsection{First visualization with simple models}\label{sec:firstvisu}

We present the series of yearly maxima in Figure \ref{first_fig} where we introduce 3 \textbf{models for the trend} :


\begin{figure}[!htb]
	\centering\includegraphics[width=.8\linewidth]{gg12.pdf}\caption{Yearly maxima together with three first models that represent the trend. Note that shaded grey area around the linear regression fit represent its $95\%$ pointwise confidence interval on the fitted values (i.e., $\pm 1.96\times \sigma_{pred}(\text{year})$ while blue dotted lines are prediction intervals taking also into account the prediction uncertainty.}%i.e. there is $95\%$ confidence that the "true" (population) regression line lies within the shaded region.}
\label{first_fig}
\end{figure}


\begin{itemize}
\item \emph{Linear regression} is a \textbf{parametric} fit. We remark that it is slightly but significantly increasing over time ($\text{p-value}\approx 10^{-5}$). From this model, one rough interpretation we can deduce is that each year we expect the annual maximum temperature to increase by $\hat{b}=0.025^{\circ}c$ \  ($\hat{\sigma}_{\hat{b}}=0.005)$.

\item \emph{Local Polynomial regression } or LOESS is a \textbf{nonparametric} fit. It tells us that the yearly maxima process is rather "smooth". The drop in the series visible around years $1950$ to $1975$ is probably due to noise rather than a real decrease or freezing of the maximum temperatures at this time. Moreover, it disappears if we change the parameter controlling the degree of smoothing. We will assess that more formally in the \hyperref[sec:splines]{next Section} with another nonparametric model.

\item \emph{Broken-linear regression} is a \textbf{parametric} fit. We wanted to emphasize visually the trend difference between the period [1901-1975] and [1976-2016]. These two periods have been chosen arbitrarily and we remark that period [1901-1952] has also a very high positive slope. We will study that in more details in \hyperref[sec:splines]{next Section}.
\end{itemize}
These different models give us insights on the process we will study throughout the rest of this thesis. We will not develop further the results of each model as this is not the subject of this thesis and we are more interested in assessing this trend and its statistical significance over time with a more interesting method.


\subsection{Deeper Trend Analysis : Splines derivatives in GAM}\label{sec:splines}

Apart from the significant pointwise increasing linear trend, we did not find concrete results. Additionally, we see in the series in Figure \ref{first_fig} that a linear trend to the entire series could be considered as too restrictive. Regarding the broken-linear trend for example, we would like to assess if there is indeed a difference in the slope of the trend over some time periods. 

We assume the reader knows about \emph{Generalized Additive Models} (GAM) developed by \citet{hastie_generalized_1986}. We also assume the reader knows about \emph{penalized splines} and \emph{splines smoothing}. Theoretical explanations of these concepts can be found in \citet[chapter 3, 6 and 11]{ruppert_semiparametric_2003} which will be the reference book of this section. Replacing covariates with smooth functions such as smoothing splines will result in a more flexible nonparametric model. We will also follow the simulation-based Bayesian approach of \citet{marra_coverage_2012} that compares coverage properties of intervals.


\addcontentsline{toc}{subsubsection}{Pointwise vs Simultaneous intervals}
\subsubsection*{Pointwise or Simultaneous confidence intervals ?}

For this analysis, we deem important to differentiate the two types of intervals for a better understanding of the actual meaning of "confidence interval". A special example is the grey area around the linear fit in Figure \ref{first_fig} which represents a $95\%$ interval for the regression line (and not for the data), taking into account variability in the data. If we repeatedly sample over and over and take the predicted values, then the new linear fit will be in the grey zone approximately $95\%$ of the time.

Let $\mathcal{X}$ denote the set of $x$ values of interest, i.e. $\mathcal{X}=[1901,2016]$ in our case and $f(\cdot)$ is the model of interest, e.g. the linear regression, or the GAM model that we will fit. Hence, we define 
\begin{itemize}
	\item A \textbf{pointwise} $100(1-\alpha)\%$ confidence interval $\Big\{\big[L(x),U(x)\big] \ : \ x\in \mathcal{X}\Big\}$ approximately satisfies
	\begin{equation}
	\text{Pr}\Big\{L(x)\leq f(x)\leq U(x)\Big\}\geq 1-\alpha, \qquad \forall x\in\mathcal{X}.
	\end{equation}
	
	\item A \textbf{Simultaneous} $100(1-\alpha)\%$ confidence interval must satisfy 
	\begin{equation}
	\text{Pr}\Big\{L(x)\leq f(x)\leq U(x), \ \forall x\in\mathcal{X}\Big\}\geq 1-\alpha.
	\end{equation}
\end{itemize} 
In the linear regression example of Figure \ref{first_fig}, from the pointwise confidence intervals we can say that, $f(1920)$ has $95\%$ chance to lie within $[29.8,\ 31.1$] and $f(2016)$ has also $95\%$ to lie within $[32.1, \ 33.6]$ but it is a fallacy to say that both are contained in these intervals at the same time with $95\%$ confidence. The corresponding simultaneous intervals (blue dotted lines) are quite larger, $[26.4, \ 34.3]$ and $[36.8, \ 28.8]$ for years $1920$ and $2016$ respectively.

For the interested reader, theoretical details on how to mathematically derive simultaneous intervals are available in \citet[pp.142-144]{ruppert_semiparametric_2003}.


\addcontentsline{toc}{subsubsection}{Methodology}
\subsubsection*{Methodology}

\begin{itemize}
\item We have fitted a simple GAM model on the annual maxima relying on the \texttt{mgcv} package from \citet{maindonald_data_2006}.

\item  Figure \ref{fig:acfresgam1} in \hyperref[app:fig]{Appendix \textbf{\ref{app:fig}}} shows the correlation structure of the serial normalized residuals. As selection of a model for the residuals is not trivial from this figure, we fitted several time series models for the residuals. It is not necessary to consider too complex models, therefore 2 additional degrees of freedom are sufficient. Results are in Table \ref{table:gamresid} in \hyperref[app:fig]{Appendix \textbf{\ref{app:fig}}} where we see that BIC will prefer the independent model for the residuals but the AIC won't. Parsimony being important, we chose the independent model to draw plots with simultaneous intervals but we will also keep MA(1) for comparisons. Likelihood ratio tests validated our choice. Diagnostics of the model presented in Figure \ref{fig:diagnogam} in \hyperref[app:fig]{Appendix \textbf{\ref{app:fig}}} are correct.

\item As we took a Gaussian (identity) link, our model can hence be written as

\begin{equation}\label{eq:gam}
Y_{\text{GAM}}(\text{year}) = \alpha + f_{\text{(k)}}(\text{year})+\epsilon , \qquad\quad \epsilon\sim\text{WN}.
\end{equation}
where $f$ is modelled by smoothing splines, $\epsilon$ can be MA(1) and $Y$ represent annual TX. We set the dimension of the basis $k$ for the spline to $20$ to ensure a reasonable degree of smoothness as there is modest amount of non-linearity in the series and we did not perform cross-validation as it will not influence the results that much. 

\item  To account for uncertainty, we simulated $M = 10^4$ draws (quite fast) from the posterior of the GAM model in (\ref{eq:gam}). Hence, the confidence intervals (or bands) that we compute will be of the form of Bayesian credible intervals,  discussed in \hyperref[bayes_cred_int]{Section \textbf{\ref{bayes_cred_int}}} and which is highlighted by \citet{marra_coverage_2012}.
$50$ posterior draws are displayed in Figure \ref{fig:post_draws} %let in \hyperref[app:fig]{appendix \ref{app:fig}}
 displaying pointwise (in yellow) and simultaneous (in red) intervals. We point out that pointwise intervals are too narrow.
\end{itemize}

\begin{figure}[!htb]
	\centering	\includegraphics[width=.75\linewidth]{post_draws.pdf}\caption{Draws from the posterior distribution of the model. Notice the large uncertainty associated to the posterior draws (grey lines).  The coverages are calculated for $M=10^4$ simulations. }\label{fig:post_draws}
\end{figure}

\subsubsection*{Coverage Analysis} 

We would like to highlight the inadequacy of the pointwise intervals. Hence, we computed posterior draws of the fitted GAM and we counted how many draws lied within each interval.

\begin{table}[!htbp] \centering 
  \caption{Proportion of the $M$ posterior simulations which are covered by the confidence intervals} \label{tab:cov} 
\begin{tabular}{@{\extracolsep{5pt}}lccccc} 
\toprule	
\vspace{-.1cm}\\[-1.8ex] 
\textbf{Coverage at $95\%$} & \multicolumn{1}{c}{$M=20$} &  \multicolumn{1}{c}{$M=100$} & \multicolumn{1}{c}{$M=10^3$} & \multicolumn{1}{c}{$M=10^5$} \vspace{.1cm} \\ 
\midrule	
\underline{Pointwise} & $40\%$ & $63\%$ & $61.1\%$ & $59.463\%$ \\
\underline{Simultaneous} & $80\%$ & $91\%$ & $94.9\%$ & $95.019\%$  \\
\bottomrule	
\end{tabular} 
\end{table}
We clearly see that the coverages indeed converge to the true value $95\%$ of the confidence level for the simultaneous intervals when $M$ becomes very large, while it converges rather to $\approx 60\%$ for the pointwise intervals.

Shiny application has been build from this graph for better visualization of the impact of the number of simulations on the confidence intervals and how their coverage varies, both visually and quantitatively.
Finally, note that the results are similar if we do the experiment on the first derivatives $f^{'}(\cdot)$ of the splines rather than on the splines themselves.


\addcontentsline{toc}{subsubsection}{Final Results}
\subsubsection*{Final Results}

We see two significant increases in the trend fitted by GAM in Figure \ref{fig:center_gam} in \hyperref[app:fig]{Appendix \textbf{\ref{app:fig}}} when we do not use  simultaneous intervals. The decrease pointed out in Figure \ref{first_fig} with LOESS between $1945$ and $1970$ is hence not significant and probably due to randomness.

Some remarks from the two plots in Figure \ref{fig:derivsplines}, considering now splines derivatives :

\begin{figure}[!htb]
	\includegraphics[width=.99\linewidth]{splines.pdf}\caption{Plots of the first derivative $f_{(20)}^{'}(\text{year})$ of the estimated splines on the retained GAM model. Grey area represents $95\%$ confidence bands. Sections of the spline where the confidence interval does not include zero are indicated by thicker lines. }\label{fig:derivsplines}
\end{figure}

\begin{itemize}
\item Together with the series in Figure \ref{first_fig}, we can make the link between the trend behavior of the series and the splines' first derivatives which accurately models the slope behavior of the trend. 
\item We notice the increasing trend with decreasing slope, i.e. $ f_{(20)}^{''}(\text{year})<0$, until year $\approx 1945$ where it becomes negative. Then the slope increases and the trend starts to increase again near 1962. This upward trend in the series of annual maxima for this last period brings light to the climate change we are currently facing. However, we see that we are now at a "critical point" since several years, i.e. $f_{(20)}^{''}(\text{year})\approx 0$, meaning that the trend is likely linear. 

\item  Whereas the pointwise confidence interval includes significant regions, i.e. 0 is not included in the interval, the simultaneous interval that is accounting for the increase in uncertainty, has no significant regions. Therefore, we cannot conclude that there has been any significant increase in the annual maximum temperatures in any period.

\end{itemize}


\section{Comments and Structure of the Analysis }

 We have found that there is indeed an upward (linear) trend for the series of yearly maxima but it is not significant once applied the correction for simultaneous intervals.
Hence, a nonstationary analysis is worth to be continued.
After this introductory analysis, we will go through with the specific subject of this thesis, the extreme value analysis. Hence, next chapter will first present a stationary GEV analysis and then allow nonstationarity in various ways.


\paragraph*{POT} 
As you have seen, the analysis in POT or in GEV involves different methods and different subsets of data. In this text, we will not display the results of the POT analysis for readability purposes. Henceforth, we will focus on the GEV analysis only. But note that the POT analysis is available on the same \href{https://github.com/proto4426/PissoortThesis}{repository} as presented above.
\hyperref[appgit]{Appendix\textbf{ \ref{appgit}}} summarizes its contents.


Moreover, we have also conducted some analysis by dividing the dataset by seasons, by hot or cold months (July-August or January-February), etc. We have also analyzed annual minimum temperatures and found a less pronounced trend than for maxima.
Interesting comparisons are also available on the \href{https://github.com/proto4426/PissoortThesis}{repository}, but we will focus on a GEV analysis on yearly maxima.