\documentclass[11pt,a4paper,openany, twosided]{book}
\makeatletter
\title{MEMOIRE}
\makeatletter
\newcommand{\crossout}[1]{%
\begingroup
\settowidth{\dimen@}{#1}%
\setlength{\unitlength}{0.05\dimen@}%
\settoheight{\dimen@}{#1}%
\count@=\dimen@
\divide\count@ by \unitlength
\count0=20 \count4=\count@\\
\loop
\count2=\count0 % keep a copy
\divide\count2\count4 \multiply\count2\count4
\ifnum\count2<\count0
\advance\count0 -\count2 % the remainder
\count2=\count0
\count0=\count4
\count4=\count2
\repeat
\count0=20 \divide\count0\count4
\count2=\count@ \divide\count2\count4
\begin{picture}(0,0)
\put(0,0){\line(\count0,\count2){20}}
\put(0,\count@){\line(\count0,-\count2){20}}
\end{picture}%
#1%
\endgroup
}
\makeatother
\newcommand\independent{\protect\mathpalette{\protect\independenT}{\perp}}
\def\independenT#1#2{\mathrel{\rlap{$#1#2$}\mkern2mu{#1#2}}}

\makeatletter % No page counter for \part
\renewcommand\part{%
	\if@openright
	\cleardoublepage
	\else
	\clearpage
	\fi
	\thispagestyle{empty}%   % Original »plain« replaced by »emptyx
	\if@twocolumn
	\onecolumn
	\@tempswatrue
	\else
	\@tempswafalse
	\fi
	\null\vfil
	\secdef\@part\@spart}
\makeatother

\usepackage{newclude} % delete \newpage with \include*{}
\usepackage{booktabs,multirow}% 

\usepackage[dvipsnames]{xcolor}
\usepackage[utf8]{inputenc} 
\usepackage[T1]{fontenc} 
%\usepackage[sc]{
\usepackage{lmodern}
%\usepackage[nomath]{kpfonts} % coment to return to lpmodern complet
\usepackage{libertine}%  serif and sans 
%\usepackage[scaled=0.85]{beramono}%% mono
\usepackage{setspace}
\linespread{1.05}

\usepackage[english]{babel}
\usepackage[a4paper]{geometry}
\usepackage{graphicx}
\usepackage[font=small,labelfont={bf,it},textfont=it]{caption}
\usepackage{csquotes}
\usepackage{subcaption}
\usepackage{pdfpages}
\usepackage[flushleft]{threeparttable}
\usepackage{multirow}
\usepackage{footnote}
\makesavenoteenv{array}
\usepackage[bottom]{footmisc}
\usepackage{tabularx, booktabs}
\newcolumntype{Y}{>{\centering\arraybackslash}X}
\usepackage{listings}
\newcommand{\cmark}{\ding{51}}%
\newcommand{\xmark}{\ding{55}}%
\newcolumntype{C}[1]{>{\centering\let\newline\\\arraybackslash\hspace{0pt}}m{#1}}
\usepackage{longtable}

\usepackage{sidecap}
\usepackage{pdflscape}
\usepackage{wrapfig}
\usepackage{listings}
\usepackage{babel,blindtext}
\usepackage{lipsum}
%\usepackage{slashbox}
\usepackage{dashrule}
\usepackage{hhline}
\usepackage{color}
%\usepackage{subfig}
\usepackage{graphicx}


\usepackage{amsmath} % Les math
\makeatletter
\let\reftagform@=\tagform@
\def\tagform@#1{\maketag@@@{(\ignorespaces\textcolor{blue}{#1}\unskip\@@italiccorr)}}
\renewcommand{\eqref}[1]{\textup{\reftagform@{\ref{#1}}}}
\makeatother
\usepackage{amssymb} % Symboles math
\usepackage{esvect} % Vecteurs
\usepackage[amssymb]{SIunits}
\usepackage{eurosym}
\usepackage{mathrsfs}
\usepackage{dcolumn}
\usepackage{pifont}
\usepackage{amsthm}

\usepackage[breaklinks,colorlinks]{hyperref}
\hypersetup{
colorlinks=true,linkcolor=blue,%citecolor=CornflowerBlue,naturalnames=true,hypertexnames=false,
	filecolor=magenta,      
	urlcolor=cyan,
	citecolor=ForestGreen,
}
\usepackage{etoolbox}


\usepackage[ruled,vlined]{algorithm2e}
\usepackage{mathtools}

\newcommand{\expect}{\operatorname{E}\expectarg}
\DeclarePairedDelimiterX{\expectarg}[1]{[}{]}{%
	\ifnum\currentgrouptype=16 \else\begingroup\fi
	\activatebar#1
	\ifnum\currentgrouptype=16 \else\endgroup\fi
}
\newcommand{\innermid}{\nonscript\;\delimsize\vert\nonscript\;}
\newcommand{\activatebar}{%
	\begingroup\lccode`\~=`\|
	\lowercase{\endgroup\let~}\innermid 
	\mathcode`|=\string"8000
}
\newlength{\mylen}  % change size of items
\setbox1=\hbox{$\bullet$}\setbox5=\hbox{\normalsize$\bullet$}
\setlength{\mylen}{\dimexpr0.5\ht1-0.5\ht2}
\makeatletter
% Patch case where name and year are separated by aysep
\patchcmd{\NAT@cite}
{\@cite\NAT@hyper@{%
		\NAT@nmfmt{\NAT@nm}%
		\hyper@natlinkbreak{\NAT@aysep\NAT@spacechar}{\@citeb\@extra@b@citeb}%
		\NAT@date}}
{\@cite\NAT@nmfmt{\NAT@nm}%
	\NAT@aysep\NAT@spacechar\NAT@hyper@{\NAT@date}}{}{}

% Patch case where name and year are separated by opening bracket
\patchcmd{\NAT@cite}
{\@cite\NAT@hyper@{%
		\NAT@nmfmt{\NAT@nm}%
		\hyper@natlinkbreak{\NAT@spacechar\NAT@@open\if#1\else#1\NAT@spacechar\fi}%
		{\@citeb\@extra@b@citeb}%
		\NAT@date}}
{\@cite\NAT@nmfmt{\NAT@nm}%
	\NAT@spacechar\NAT@@open\if#1\else#1\NAT@spacechar\fi\NAT@hyper@{\NAT@date}}
{}{}

\makeatother

\usepackage{datetime} 
\usepackage{bm}
\usepackage{scalerel,stackengine}
\usepackage{pbox}
\usepackage{vmargin}            
\setmarginsrb{2,5cm}{2,7cm}{2,5cm}{2,5cm}{0cm}{1cm}{0cm}{1cm}
\rmfamily
%\DeclareFontShape{T1}{lmr}{b}{sc}{<->ssub*cmr/bx/sc}{}
%\DeclareFontShape{T1}{lmr}{bx}{sc}{<->ssub*cmr/bx/sc}{}

\DeclareMathOperator*{\argminA}{arg\,max} % Jan Hlavacek
\DeclareMathOperator*{\argminB}{argmax}   % Jan Hlavacek
\DeclareMathOperator*{\argminC}{\arg\max}   % rbp

\newcommand{\argmaxD}{\arg\!\max} % AlfC

\newcommand{\argminE}{\mathop{\mathrm{argmin}}}          % ASdeL
\newcommand{\argminF}{\mathop{\mathrm{argmin}}\limits} 

\usepackage{fancyhdr}
\usepackage{afterpage}

\fancypagestyle{plain}{\fancyhf{}}
\pagestyle{fancy}
\fancyfoot{}
\cfoot{\thepage}
\fancyhead[RO,LE]{\thepage}
\fancyhead[LO]{\leftmark}
\fancyhead[RE]{\rightmark}
\renewcommand{\headrulewidth}{.6pt}

\iffalse
\copypagestyle{memoirStylePages}{headings}
\makerunningwidth{memoirStylePages}{\textwidth}
\makeheadrule{memoirStylePages}{\textwidth}{\normalrulethickness}
\pagestyle{memoirStylePages}
\fi


\pagenumbering{roman}
\newcommand{\bib}{\par\noindent\hangindent=0.5 true cm \hangafter=1}

\makeatletter  % rm pg number in bottom
\let\@oddfoot\@empty
\let\@evenfoot\@empty
\makeatother
\usepackage{appendix}
%\usepackage[style=apa,backref=true,backend=biber,natbib=true,hyperref=true]{biblatex} 
\let\origappendix\appendix % save the existing appendix command
\renewcommand\appendix{\clearpage\pagenumbering{Roman}\origappendix}


%\addbibresource{zotero.bib.bib}
\usepackage[sort, square, comma]{natbib}
%\usepackage{natbib}
\bibliographystyle{abbrvnat}
\setcitestyle{authoryear, open = {[}, close={]} }

%\setcitestyle{square}
%\setcitestyle{authoryear, open={[},close={)]}}
%\usepackage{apacite}
%\usepackage{cite}
%\usepackage[backend=biber]{biblatex}
\usepackage{bbm}
\usepackage{enumitem}
\usepackage{minitoc}

\newtheorem{theorem}{Theorem}[chapter]
\newtheorem{corollary}{Corollary}[theorem]
\newtheorem{lemma}[theorem]{Lemma}
%\theoremstyle{definition}
\newtheorem{definition}{Definition}[chapter]
%\renewcommand*{\proofname}{\boxed{Proof}}
\usepackage{thmtools}

\newcommand*{\QEDB}{\hfill\ensuremath{\triangle}}%
\newcommand*{\QEDA}{\hfill\ensuremath{\blacksquare}}%

\AtEndEnvironment{theorem}{\null\hfill\qedsymbol}%
\AtEndEnvironment{definition}{\null\hfill\QEDB}%
%\declaretheorem[style=definition,qed=$\triangle$,sibling=definition]{definition}
%\declaretheorem[style=definition,qed=$\triangle$,sibling=definition]{theorem}
\newtheorem{exe}{Theorem}
\setcounter{exe}{-1}

\usepackage{multirow,bigdelim}
\usepackage{tcolorbox}
\usepackage{adjustbox}
\usepackage{dashbox}
\usepackage{todonotes}

\newcommand{\abbrlabel}[1]{\makebox[3cm][l]{\textbf{#1}\ \dotfill}}
\newenvironment{abbreviations}{\begin{list}{}{\renewcommand{\makelabel}{\abbrlabel}}}{\end{list}}


\usepackage{tikz}
\usetikzlibrary{matrix,chains,positioning,decorations.pathreplacing,arrows}
%\defbeamertemplate{itemize item}{tikzarrow}{\tikz{\node[single
%		arrow,scale=0.2,inner sep=2ex,fill] at (0,0) {};}}

\tikzset{myarrow/.style={
		draw,
		fill=Emerald,
		single arrow,
		minimum height=5.5ex,
		single arrow head extend=1ex
	}
}
\newcommand{\arrowright}{%
	\tikz [baseline=-1ex]{\node [myarrow,rotate=0] {};}
}
\setlength{\parskip}{.3em}

\usepackage{afterpage}

\newcommand\blankpage{%
    \null
    \thispagestyle{empty}%
    \addtocounter{page}{-1}%
    \newpage}


\setcounter{tocdepth}{3}
\setcounter{secnumdepth}{4}
\setcounter{minitocdepth}{4}

% Chapter head 
\makeatletter
\def\thickhrulefill{\leavevmode \leaders \hrule height 1ex \hfill \kern \z@}
\def\@makechapterhead#1{%
  {\parindent \z@ \raggedright
    \reset@font
    \hrule
    \vspace*{10\p@}%
    \par
    \Large \scshape \@chapapp{} \Huge\bfseries \thechapter
    \par\nobreak
    \vspace*{10\p@}%
    \hrule
    \par
    \vspace*{1\p@}%
    \hrule
    %\vskip 40\p@
    \vspace*{20\p@}
    \Huge \bfseries #1\par\nobreak
    \vskip 70\p@
  }}

\begin{document}
\thispagestyle{empty}


\begin{center}
	{\LARGE    UNIVERSITE CATHOLIQUE DE LOUVAIN} \\
	\vspace{0.2cm}
	{\large FACULTE DES SCIENCES} \\
	\vspace{0.2cm}
	{\Large ECOLE DE STATISTIQUE, BIOSTATISTIQUE \\
		\vspace{0.1cm}
		ET SCIENCES ACTUARIELLES }\\
\end{center}
\vfill

%\begin{figure}[H]
\begin{center}  \hspace*{-10mm} 
	\includegraphics[height = 3.5cm]{lsba.jpg}
\end{center}
%\end{figure}

\thispagestyle{empty}

\vfill


\begin{center}
	{\Large \textsc{Temporal analysis of the evolution of extreme values using climatological data}}
	
\end{center}


\vfill

\begin{center}[]
	\begin{minipage}[c]{.45\linewidth}
		\begin{tabular}{ll}
			\vspace{0.2cm}
			Promoteur : & Johan \textsc{Segers} \\
			\vspace{0.2cm}
			Lecteurs : & Anna \textsc{Kiriliouk}\\
			\vspace{0.2cm}
			& Michel \textsc{Crucifix}
		\end{tabular}
	\end{minipage} \hfill
	\begin{minipage}[c]{.45\linewidth}
		\begin{tabularx}{\linewidth}{p{\textwidth}}
			Mémoire présenté en vue de \hbox{l'obtention} du \\   \vspace{0.1cm} Master en statistiques, orientation générale \\   \vspace{0.1cm} par :
			\textbf{Antoine \textsc{Pissoort}}
		\end{tabularx}
	\end{minipage}
\end{center}

\vspace{1,5cm}



\begin{center}
	{\large Juin 2017}
\end{center}
\thispagestyle{empty}
\newpage


%\centering
\topskip0pt
\vspace*{\fill}
%\addcontentsline{toc}{chapter}{Abstract}%
\section*{\centering Abstract}
\begin{tcolorbox}%[colback=SeaGreen!75!white]
This thesis aims to analyse extreme temperatures from Uccle and assess their nonstationarity. trend in the location parameter of the temperatures assessing the climate warming.   
After having proven it by hand of splines's deriatives that a correction for simultaneous intervals led to non-significant changes , we will do it by hand of the Extreme Value Theory (EVT)  that there is indeed an upward trend in the location parameter of the temperatures assessing the climate warming. First of all, we will do introductory analysis of the trend... and we will discover that..  The analysis will focus on yearly maxima and hence we will go through with EVT by defining and presenting the usual methods such as GEV. 

Regarding the computations, we took advantage of a high-level language (c++) to make our analysis efficient and also made use of parallel computing to decrease computation time for time consuming...
\thispagestyle{empty}
\end{tcolorbox}

\vspace{.3cm}

\paragraph*{Keywords} $\bullet$ Extreme Value Theory   $\bullet$ block-maxima model  $\bullet$ peaks-over-threshold method $\bullet$ trend analysis $\bullet$ Bayesian inference   $\bullet$ Generalized Additive Models with splines smoothing  $\bullet$ Neural Networks  $\bullet$ nonstationary models $\bullet$ Markov Chain Monte Carlo $\bullet$ Hamiltonian Monte Carlo $\bullet$ Parallel computing $\bullet$ R package $\bullet$ Shiny application
\vspace*{\fill}

\newpage


%\afterpage{\cfoot{\thepage}}


\newenvironment{acknowledgements}%
{\thispagestyle{empty}\null\vfill\begin{center}%
\bfseries Acknowledgements\end{center}}%
{\vfill\null}
%\addcontentsline{toc}{chapter}{Acknowledgements}%
\begin{acknowledgements}
	\textit{I would first like to thank my thesis supervisor Johan Segers for all his help and his guidance during this whole year. The repeated appointments we have had   }
	\newline
	
	\textit{I also would like to thank the "Institut Royal de Météorologie" (IRM) of Belgium for his help and his guidance but also for his provided quality datasets.}
	\newline
	
	\textit{A bit less usual, I would like to thank the open source community such as R or LaTeX can benefit. For me and for a number of student, it has been a nonnegligable source of ideas and the most efficient source of learning.  }
			\newline
	\textit{Finally, I want to thank my family and my friends, but also Bernadette for her support and all the time I have spent  writing in her room for my thesis but also during my whole academic studies.}
	
	\thispagestyle{empty}
\end{acknowledgements}


\afterpage{\blankpage}


\thispagestyle{empty}
\dominitoc
\thispagestyle{empty}
\tableofcontents
\thispagestyle{empty}
\newpage
\thispagestyle{empty}
\listoffigures
\thispagestyle{empty}
\listoftables
\thispagestyle{empty}

\newpage


\chapter*{List of Abbreviations}
%\addcontentsline{toc}{chapter}{List of Abbreviations}
\thispagestyle{empty}
For convenience, we place a list of all the abbreviations we will use in the text. However, these will always be defined in their first occurrence in the text.\\

\begin{center}
\begin{abbreviations}
	\item[DA] Domain of Attraction
	\item[df]\label{df}  (cumulative) distribution function
	\item[EVI] Extreme Value Index ($\xi$)
	\item[EVT] Extreme Value Theory
	\item[GEV] Generalized Extreme Value
	\item[GML] Generalized Maximum Likelihood
	\item[GPD] Generalized Pareto Distribution %(function)
	\item[MCMC] Marko Chain Monte Carlo
	\item[MH] Metropolis-Hastings (algorithm)
	\item[ML] Maximum Likelihood 
	\item[MLE] Maximum Likelihood Estimator
	\item[NN] Neural Network
	\item[TN] Temperature miNimum
	\item[TX] Temperature maXimum
	
\end{abbreviations}
\end{center}


\renewcommand\labelitemi{\normalsize$\bullet$}



\afterpage{\blankpage}


\addcontentsline{toc}{chapter}{Introduction}
\chapter*{Introduction}
	%Presentation of the Analysis }
\pagenumbering{arabic}
\thispagestyle{empty}



Unlike its counterparts ( see for example credit risk analysis, financial applications,...), the Extreme Value Theory (EVT) applied on broad environemental area such as meteorological data, has strong impacts on the people lives 

An important question is still whether climate changes caused by anthropogenic activities will change the intensity and frequency of extreme events \cite{milly_climate_2008}.

The problem facing climate change evidence is the lack of past data to compare with her

For such an analysis, the number of parameters to take into account is considerable (and tend to infinity)  

To make a parallel with Chaos Theory and the well-known butterfly effect which have strong applications in weather models\newline

We anticipate that climate change to affect the extreme weather  \newline

[extremes in climate change p.347]


It has been proven that winter is becoming warmer in context of RC. (see naveau,...)


\cite{kharin_changes_2006}


"The first myth about climate extremes, which has been purported by
researchers in climatology or hydrology, among them prominent names, is that
“extremes are defined as rare events” or similar. This myth is debunked by a simple
bimodal PDF (Fig. 6.12a). The events sitting in the tails of that distribution are not
rare" \cite[pp.257]{mudelsee_climate_2014}
\newline

Until now, studies on climate extremes in Europe have usually had a strong national signature
%(e.g. Herzog and Muller-Westermeier, 1997; Brunetti ¨ et al., 2001)
, or have had to make use of either a
dataset with daily series from a very sparse network of meteorological stations (e.g. eight stations in Moberg
et al. (2000)) or standardized data analysis performed by different researchers in different countries along the
lines of agreed methodologies (e.g. Brazdil et al., 1996; Heino et al., 1999) \cite{klein_tank_daily_2002}


Extrapolation !!!! See p154 [statistical analysis of extreme book]



Voir effet de l'ilot de chaleur --> urbanisation sur les tempés !

---> artificial warming on cities stations which were not(less) urbanized 100 years ago.

[In this thesis, efforts have been made to use power of (hyper)references into the 
text. While this not (yet ?...) usable in printed versions, the reader may feel more 
comfortable in a numeric version to more easily handle the vast amount of sections, 
equations, references, etc... and the links that are made between them.]


We can summarize the research question of this thesis as the following : 

\begin{itemize}
\item 
\end{itemize}


In this part, we will make use of general methods to assess if there is indeed a trend in the maximum temperatures

There are two main approaches in EVT, the block-maxima  and the peaks-over-threshold approach (see 
\hyperref[sec::2]{Section \textbf{2}}) yielding to different extreme value distribution. 
The former aims at while the latter models the (...)


In \hyperref[sec::1]{Chapter \textbf{1}} we will present the method of block-maxima and derive the Generalized Extreme Value (GEV) distributions. In \hyperref[sec::2]{Chapter \textbf{2}} we will .
In \hyperref[sec::3]{Chapter \textbf{3}}
In \hyperref[sec::4]{Chapter \textbf{4}} we will ..

Finally, we notice that this thesis will concentrate on the block-maxima (GEV) methods (see \hyperref[sec::1]{Chapter \textbf{1}}), i.e. to data relating to maxima over a period of 1 year. For example here in the introduction, or for the Bayesian analysis in \hyperref[sec::bayesian]{Chapter \textbf{5}}.
\thispagestyle{empty}



\begin{center}
\part{Theoretical Framework : Extreme Value Theory}
\end{center}
%\pagestyle{fancy}


\setcounter{mtc}{1}
\chapter{Method of Block Maxima} \label{sec::1}
\vspace{-1cm}
\minitoc \thispagestyle{empty}
 \vspace{1cm}

This chapter introduce the basics of EVT by considering the \textit{block-maxima} approach. After defining useful concepts in 
\hyperref[sec::1.1]{Section \textbf{1.1}} to introduce the emergence of this theory, we will get into the leading theorem of EVT in \hyperref[sec:extrtypethm]{Section \textbf{\ref{sec:extrtypethm}}}. \hyperref[sec::appconcrete]{Section \textbf{\ref{sec::appconcrete}}} will present some mathematical applications and \hyperref[sec:mda]{Section \textbf{\ref{sec:mda}}} conditions  of this theorem in order to visualize the implications of the extremal theorem and the characterizations of the underlying distributions. \hyperref[rlgev]{Section \textbf{\ref{rlgev}}} will introduce key concepts of inference in EVT which will be presented in a general (and frequentist) way in  \hyperref[sec::gevinfernce]{Section \textbf{\ref{sec::gevinfernce}}}. Finally, \hyperref[sec:diag]{Section \textbf{\ref{sec:diag}}} provides some tools to assess the accuracy of the fitted model.

This chapter is mostly based on \citet[chap.3]{coles_introduction_2001}, \citet[chap.2]{beirlant_statistics_2006} and \citet[chap.1-4]{reiss_statistical_2007}, and other relevant articles.

\newpage
\include*{chap1}



\chapter{Peaks-Over-Threshold Method}\label{sec::2}
\vspace{-1cm}
\minitoc \thispagestyle{empty}
 \vspace{1cm}

This new chapter will focus on another major approach of EV models by modeling only the excess over a certain threshold. This approach is very popular in practice as it can handle all the extremes with more flexibility and not only the maximum of one block. From this fact, numerous techniques have emerged and we will quickly navigate the main ideas.

In \hyperref[sec::2.1]{Section \textbf{\ref{sec::2.1}}}, we will intuitively introduce the approach that will help us 
to formally characterize the resulting distribution of interest in \hyperref[sec:charac_gpd]{Section \textbf{\ref{sec:charac_gpd}}}. Then, \hyperref[sec:rl_gpd]{Section \textbf{\ref{sec:rl_gpd}}} will present the concept of return levels applied to this concept. Next, we will assess the maximum temperature threshold in a statistical sense in \hyperref[sec:thresh_selec]{Section \textbf{\ref{sec:thresh_selec}}} by reviewing available methods, regardless common suggested thresholds of $25^{\circ} c$ or $30^{\circ} c$ from meteorologists.

Unfortunately, the resulting analysis of this chapter will not be present in \hyperref[part:xp]{Part \textbf{\ref{part:xp}}} because it would have made the text become too voluminous. But empirical results (code and html reports) will remain available in the \href{https://github.com/proto4426/PissoortThesis/}{repository}\footnote{\url{https://github.com/proto4426/PissoortThesis/}} which structure is explained in \hyperref[appgit]{Appendix \textbf{\ref{appgit}}}. Moreover, \emph{point process} will not be covered for the same reason but we remind that this is a powerful and flexible method that summarizes the two techniques considered in the two first chapters, and it should then not be missed. 

This chapter is mainly based on \citet[chap.4 and 7]{coles_introduction_2001}, \citet[chap.4]{beirlant_statistics_2006}, \citet[chap.5]{reiss_statistical_2007} and 
\citet[chap.5]{embrechts_modelling_2011},
 and other relevant articles.

\newpage
\include*{chap2}



\chapter{Relaxing The Independence Assumption}\label{sec::3}
\vspace{-1cm}
\minitoc\thispagestyle{empty}
 \vspace{1cm}
 
In most environmental applications, the independence assumption made in the first chapters is questionable and never completely fulfilled. From hydrological process as stated in \citet{milly_climate_2008} to temperature data where climate warming is a famous issue, such theoretical assumptions that have been previously made are not sustainable in practice. First, 
\hyperref[sec:statio]{Section \textbf{\ref{sec:statio}}} will provide tools that enable EV models to hold in presence of a limited long-range dependence.
\hyperref[nstatio]{Section \textbf{\ref{nstatio}}} will allow EV modeling under nonstationary processes allowing us for instance to study the possible increasing behavior of the maximum temperatures. Whereas \hyperref[sec:returnlvlnstatio]{Section \textbf{\ref{sec:returnlvlnstatio}}} redefined return levels under those less restricted cases,   \hyperref[sec:gevcdn]{Section \textbf{\ref{sec:gevcdn}}} will introduce a new flexible way to model nonstationary extremes under a redefined Multi Layer Perceptron framework providing inference techniques that minimize chance of overfitting.


This chapter is mostly based on \citet[chap.5-6]{coles_introduction_2001}, \citet[chap.10]{beirlant_statistics_2006} and \citet[chap.7]{reiss_statistical_2007}, and other relevant articles.
\newpage
\include*{chap3}




\chapter{Bayesian Extreme Value Theory}\label{sec::bayesian}
\vspace{-1cm}
\minitoc\thispagestyle{empty}
 \vspace{1cm}
 
see evdbayes pdf package r

Attention : $\pi$ ou $f$ ?????? 

We let useful relevant tools regarding bayesian inference in \hyperref[bayesapp]{Appendix \textbf{\ref{bayesapp}}}

This chapter is mostly based on \citet[sec. 9.1]{coles_introduction_2001}, \citet[chap.11]{beirlant_statistics_2006} and \citet[chap.1-4]{reiss_statistical_2007}, and other relevant articles.

\newpage
\include*{chap4}





\part{Experimental Framework : Extreme Value Analysis of Maximum Temperatures}\label{part:xp}
\thispagestyle{empty}

%In this part, we will focus on the application of the methods seen during the theoretical part. 

\chapter{Introduction to the Analysis}\label{chap:introana}
\vspace{-1cm}
\minitoc \thispagestyle{empty}
 \vspace{1cm}
 
Since we have theoretically defined useful concepts in EVT, we will now introduce the practical analysis of this thesis that consist of analyzing daily maximum temperatures in Uccle from 1901 to 2016.  
We already showed the distribution of these data in Figure \ref{fig:pot_plot} to introduce the POT approach. This chapter will present an introductory analysis of the annual maxima of these data. We chose annual maxima in order to later provide a convenient GEV modeling. We will then first use several techniques to analyze these data before going further into the EV models in the following chapters. 

After a brief presentation of the repository which contains the R package created for this thesis and the structure for the scripts that contains the code that created all the analysis, we will briefly present the shiny applications created to enhance visualizations. Then, \hyperref[sec:presuccle]{Section \textbf{\ref{sec:presuccle}}} will present data and their source(s). \hyperref[sec:firstana]{Section \textbf{\ref{sec:firstana}}} will first describe data and compare models to represent the data before going further on the trend analysis with splines derivatives in a Generalized Additive Model to assess significance of the trend with correction for simultaneous tests and provide first aspects in the issue of Climate Warming without using extreme models. 


This chapter will then not explicitly use techniques presented in the previous chapters. It will be mostly based on \citet{ruppert_semiparametric_2003} for the model presented in \hyperref[sec:splines]{Section \textbf{\ref{sec:splines}}}. 
  

\newpage
\include*{chap5}



\chapter{Analysis in Block Maxima}\label{sec:anagev}
\minitoc \thispagestyle{empty}
 \vspace{1cm}
 
 This chapter introduces the core subject of this thesis, the stationary and nonstationary GEV analysis for the annual maximum temperatures in Uccle. It will use concepts borrowed from \hyperref[sec::1]{Chapter \textbf{\ref{sec::1}}} and \hyperref[sec::3]{Chapter \textbf{\ref{sec::3}}}. POT analysis of \hyperref[sec::2]{Chapter \textbf{\ref{sec::2}}} will be mentioned but all results are let in the code. 

In \hyperref[sec:xpstatio]{Section \textbf{\ref{sec:xpstatio}}}, we will present first GEV inferences assuming stationarity. It relies on \texttt{1intro\_stationary.R} code from the \textbf{/Scripts-R/} folder of the \href{https://github.com/proto4426/PissoortThesis/}{repository}.
\hyperref[sec:xpnp]{Section \textbf{\ref{sec:xpnp}}} will introduce the nonstationary analysis with parametric models and relies on \texttt{2Nonstationary.R} code while \hyperref[sec:nnxp]{Section \textbf{\ref{sec:nnxp}}} allows for more complex models from deep archtiectures and relying on \texttt{2NeuralsNets.R} code. 

Whereas some Bayesian concepts will be used in this section such as use of \emph{prior distributions} for the EVI to limit its domain when estimating it with a likelihood-based method, or to regularize weights of Neural Networks in \hyperref[sec:nnxp]{Section \textbf{\ref{sec:nnxp}}}, we will still call all these methods as "frequentists".
A strict Bayesian analysis will be made in the \hyperref[sec:anabayes]{next Chapter \ref{sec:anabayes}} and comparisons will be provided.

\newpage
\include*{chap6}


%%%%%%%%%%%%%%%%%%%%%%%%
\iffalse
\chapter{Stationary and Nonstationary GEV Analysis}\label{sec:ananonsta}
\minitoc \thispagestyle{empty}
 \vspace{1cm}
\include*{chap8}
\fi
%%%%%%%%%%%%%%%%%%%%%%%%%


\chapter{Bayesian Analysis in Block Maxima}\label{sec:anabayes}
\minitoc \thispagestyle{empty}
 \vspace{1cm}

This analysis relies on all the codes which filenames start by "\texttt{Bayes}" from the \textbf{/Scripts-R/} folder of the \href{https://github.com/proto4426/PissoortThesis/}{repository}. All the functions created that are used and are also made available through the package are in /R/\texttt{BayesFunc.R}. Moreover,
the \texttt{/vignettes/Summary\_Bayesian.html} file in the \href{https://github.com/proto4426/PissoortThesis/}{repository} present a big part of the analysis in details.

"It is often the case that more than one model provides an adequate fit to the data. Sensitivity analysis determines by what extent posterior inferences change when alternative
models are used"   book risk analysis other section pp.2.

"The basic method of sensitivity analysis is to fit several models to
the same problem. Posterior inferences from each model can then be compared."


\newpage
\include*{chap7}




\addcontentsline{toc}{chapter}{Conclusion}
\chapter*{Conclusion}
\thispagestyle{empty}

During this thesis, we have statistically assessed the presence of a trend in the extreme temperatures in Uccle. We first detected that the trend is significative by the method of linear regression. We also discovered that the best fitted GEV model is the one with a linear trend in the location parameter.

"A key issue in applications is that inferences
may be required well beyond the observed tail of
the data, and so an assumption of stability is required:" \cite{davison_statistical_2012}

"Another approach would be to use something other than time as the covariate 
in the model. For instance, one could imagine linking temperature data directly to
CO2 level rather than time. However, linking to a climatological covariate makes
extrapolation into the future more difficult, as one would need to extrapolate the 
covariate as well. No obvious climatological covariate comes to mind for the Red
River application. " %[extremes in climate change p.111]


Timescale-uncertainty effects on extreme value analyses seem not to have been
studied yet. For stationary models (Sect. 6.2), we anticipate sizable effects on block
extremes–GEV estimates only when the uncertainties distort strongly the blocking
procedure. For nonstationary models (Sect. 6.3), one may augment confidence band
construction by inserting a timescale simulation step (after Step 4 in Algorithm 6.1) \citet[pp.262]{mudelsee_climate_2014}


!!!!! not put too much references in the text !!!!!

A project to give analysis in time live could be implemented, with values being stocked each days/year !

be prudent that all "methods" are listed (numbered) in (each) ToC

add square brackets [] for cite ? 

Careful for boldsymbols, especiallyin bayesian

PissoortThesis:: Behind our functions 

[ MOTS DEXPLICATIONS SUR TTES LES FORMULES (domain atraction condition, etc...]

\thispagestyle{empty}

Finish (Bayesian) documentation in the package.

See new graphs for stats desc. (in appendix), from workshop exam.

Create Gif animations for the github readme


%\clearpage
\addcontentsline{toc}{part}{Appendix}
\part*{Appendix}
\appendix

\include*{appendix}




\bibliographystyle{ieeetr}
\setlength{\parindent}{5em}
\setlength{\parskip}{2em}
\renewcommand{\baselinestretch}{4.0}

\bibliography{zotero1}

\end{document}